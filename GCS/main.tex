\documentclass[openany]{book}
\usepackage[utf8]{inputenc}
\usepackage{verbatim}
\usepackage[hypertexnames=false]{hyperref}
\usepackage{amstext} 
\usepackage{array}   
\newcolumntype{C}{>{$}c<{$}} 


%%%%%%%%%%%%%%%%%%%%%%%

%%%%%%%%%%%%%%%%%%%%%%%
% HOLA PACO
% ESTE ES EL ARCHIVO DE LAS DEFINICIONES ESTRUCTURALES
% VERSION 1.1 NOMÁS
%
% AUTOR ORIGINAL:
% EDUARDO (CHITO) BELMONTE GUILLAMÓN
%
% ESTE ARCHIVO ES COMUNISTA, PUEDES COMPARTIRLO SI QUIERES
%%%%%%%%%%%%%%%%%%%%%%%

%----------------------------------
%     PAQUETICOS QUE SE USAN
%----------------------------------

%--------------------------
%    PARA USAR INKSCAPE
%---------------------------
\usepackage{import}
\usepackage{hyperref}
\usepackage{xifthen}
\usepackage{pdfpages}
\usepackage{transparent}

\newcommand{\incfig}[1]{%
    \def\svgwidth{\columnwidth}
    \import{./figures/}{#1.pdf_tex}
}

\newcommand{\custincfig}[2]{%
    \def\svgwidth{#1}
    \import{./figures/}{#2.pdf_tex}
}
\newcommand{\textnexttofig}[3]{
  \begin{minipage}[l]{0.45\textwidth}
    \custincfig{#1}{#2}
  \end{minipage}
  \begin{minipage}[l]{0.45\textwidth}
    #3
  \end{minipage}
}

%%%%%%%%% FIN DEL INKSCAPE

\usepackage{parskip} % Pa parrafos wapos
\setlength{\parindent}{0.5cm} % Pa la sangría
\usepackage{graphicx} % Pa meter las imágenes
\graphicspath{{Images/}} % La ruta a las imágenes

\usepackage{tikz} % Pa dibujar cosichuelas guapas

\usepackage[spanish]{babel} % PA QUE ESTÉ EN ESPAÑOL NOMÁS

\usepackage{enumitem} % Para personalizar las LISTAS YEAH

\setlist{nolistsep} % Pa que las listas estén junticas

\usepackage{booktabs} % Esta sirve para hacer tablas fancy con multicolumns y tal pero no tengo ni puta idea de usarla

\usepackage{xcolor} % PA DEFINIR LOS COLORINES
\definecolor{turquoise}{RGB}{21,103,112} % Es un turquesica así formal
\definecolor{violet}{RGB}{ 110, 6, 187 } % Color maricón

%-------------------------------------------------
%     MÁRGENES
%-------------------------------------------------

\usepackage{geometry}
\geometry{
    top=3cm,
    bottom=3cm,
    left=3cm,
    right=3cm,
    headheight=14pt,
	footskip=1.4cm,
	headsep=10pt,
}

\usepackage{avant} % Esto es una fuente para encabezados

%\usepackage{mathptmx} % Usar simbolitos matemáticos chulos

\usepackage{microtype} % Para fuentes de maricones

\usepackage[utf8]{inputenc} % Pa los acentos

\usepackage[T1]{fontenc}

%-------------------------------------------------
% Bibliografía e índice
%-------------------------------------------------

\usepackage{makeidx} % Pa hacer un índice
\makeindex

\usepackage{titletoc}   % Para manipular la tabla de contenidos

\contentsmargin{0cm}    % Para eliminar el margen por defecto

\usepackage{titlesec} % Pa cambiar los titulos skere

\titleformat
{\chapter} % command
[display] % shape
{\centering\bfseries\Huge\normalfont} % format
{\color{turquoise}  {\normalsize\MakeUppercase{Capítulo} \thechapter }} % label
{-0.5cm} % sep
{
    \color{turquoise}
    \rule{\textwidth}{3pt}
    \vspace{1ex}
    \centering
    \setcounter{ex}{0}
    \setcounter{dummy}{0}
} % before-code
[
\vspace{-0.5cm}%
\rule{\textwidth}{3pt}
] % after-code


\titleformat{\part}
[display]
{\centering\bfseries\Huge\normalfont}
{\color{turquoise} {\normalsize \MakeUppercase{Asignatura}}}
{0pt}
{\color{turquoise}
\vspace{-0.6cm}
\rule{\textwidth}{3pt}
\vspace{1ex}
\setcounter{chapter}{0}
\setcounter{section}{0}
\setcounter{dummy}{0}
\centering
}


\titleformat{\section}
{\normalfont\Large\bfseries}{\color{turquoise}\thesection\ - }{0.5em}{}

\usepackage{fancyhdr}   % Necesario para el encabezado y el pie de página

\pagestyle{fancy}   %Para modificar los encabezados
\fancyhf{}          %Para eliminar los encabezados y pies de página por defecto.
\fancyhead[LE,RO]{\sffamily\normalsize\thepage}
\fancyfoot[C]{Ampliación de Probabilidad}
%HACER

\usepackage{amsmath,amsfonts,amssymb,amsthm,cancel} % PARA LAS MATES

%   LINEA 199, HACER CAPULLADAS

\newtheoremstyle{turquoisebox}
{0pt} %Espacio encima
{0pt} %Espacio abajo
{\normalfont} % Fuente del cuerpo
{} % Cantidad de identado
{\small\ssfamily\color{turquoise}} % Fuente en la que pone "TEOREMA"
{:} % Puntuación tras el teorema
{0.25em} %Espacio tras el teorema
{\thmname{#1}\thmnumber{#2}} %No sé si esto funciona


\newcounter{dummy}
\newcounter{ex}
\newtheorem{teoremote}[dummy]{\color{turquoise}Teorema}
\newtheorem{propositiont}{\color{turquoise}Proposición}[section]
\newtheorem{lemmat}{\color{turquoise}Lema}[section]
\newtheorem{definitionT}{\color{turquoise}Definición}[section]
\newtheorem{exerciseT}[ex]{Ejercicio}
\newtheorem{examplote}[ex]{\color{turquoise}Ejemplo}
\newtheorem{methodT}[dummy]{\color{turquoise}Método}


\RequirePackage[framemethod=default]{mdframed} % Required for creating the theorem, definition, exercise and corollary boxes

%Caja de teoremas

\newmdenv[skipabove=7pt,
skipbelow=7pt,
backgroundcolor=black!5,
linecolor=turquoise,
innerleftmargin=5pt,
innerrightmargin=5pt,
innertopmargin=5pt,
leftmargin=0cm,
rightmargin=0cm,
linewidth=3pt,
innerbottommargin=5pt]{tBox}

\newmdenv[skipabove=7pt,
skipbelow=7pt,
backgroundcolor=black!5,
linecolor=turquoise,
innerleftmargin=5pt,
innerrightmargin=5pt,
innertopmargin=5pt,
leftmargin=0cm,
rightmargin=0cm,
linewidth=1pt,
innerbottommargin=5pt]{pBox}

\newmdenv[skipabove=7pt,
skipbelow=7pt,
backgroundcolor=violet!7,
linecolor=turquoise,
innerleftmargin=5pt,
innerrightmargin=5pt,
innertopmargin=5pt,
leftmargin=0cm,
rightmargin=0cm,
rightline=false,
topline=false,
bottomline=false,
linewidth=4pt,
innerbottommargin=5pt]{mBox}

\newmdenv[skipabove=7pt,
skipbelow=7pt,
rightline=false,
leftline=true,
topline=false,
bottomline=false,
linecolor=turquoise,
innerleftmargin=5pt,
innerrightmargin=5pt,
innertopmargin=0pt,
leftmargin=0cm,
rightmargin=0cm,
linewidth=4pt,
innerbottommargin=0pt]{dBox}

\newmdenv[skipabove=7pt,
skipbelow=7pt,
rightline=false,
leftline=true,
topline=false,
bottomline=false,
backgroundcolor=black!3,
linecolor=turquoise!50,
innerleftmargin=5pt,
innerrightmargin=5pt,
innertopmargin=0pt,
innerbottommargin=5pt,
leftmargin=0cm,
rightmargin=0cm,
linewidth=4pt]{eBox}

\newmdenv[skipabove=7pt,
skipbelow=7pt,
leftline=true,
topline=false,
rightline=false,
bottomline=false,
backgroundcolor=cyan!5,
linecolor=turquoise,
innerleftmargin=5pt,
innerrightmargin=5pt,
innertopmargin=0pt,
innerbottommargin=5pt,
leftmargin=0cm,
rightmargin=0cm,
linewidth=4pt]{exBox}

\newenvironment{theorem}{\begin{tBox}\begin{teoremote}}{\end{teoremote}\end{tBox}}
\newenvironment{proposition}{\begin{pBox}\begin{propositiont}}{\end{propositiont}\end{pBox}}
\newenvironment{lemma}{\begin{pBox}\begin{lemmat}}{\end{lemmat}\end{pBox}}
\newenvironment{method}{\begin{mBox}\begin{methodT}}{\end{methodT}\end{mBox}}
\newenvironment{definition}{\begin{dBox}\begin{definitionT}}{\end{definitionT}\end{dBox}}
\newenvironment{exercise}{\begin{eBox}\begin{exerciseT}}{\hfill{\color{black}}\end{exerciseT}\end{eBox}}
\newenvironment{example}{\begin{exBox}\begin{examplote}}{\end{examplote}\end{exBox}}
\newenvironment{demonstration}{\begin{flushright}
      \color{turquoise} \textbf{Demostración}
\end{flushright}
}{\begin{flushright}
  $\square$
\end{flushright}}

\usepackage{geometry}
\geometry{
    top=3cm,
    bottom=3cm,
    left=3cm,
    right=3cm,
    headheight=14pt, 
    footskip=1.4cm,
    headsep=10pt,
}
\usepackage{graphicx}
\title{Ejercicios de Geometría de Curvas y Superficies}
\author{Paco Mora Caselles}
\date{\today}

\begin{document}


\maketitle

\chapter{Hoja 2}

\setcounter{ex}{1}

\begin{exercise}
    
    $$ f(p) = d(p,p_0) = |p-p_0| $$
    $$ F: \mathbb{R}^3 \to \mathbb{R}\ F(x,y,z) = \sqrt{(x-a)^2+(y-b)^2+(z-c)^2}$$
    F es diferenciable salvo si $ p = p_0 $, entonces $ f = F|_{S} $ es $ C^{\infty} $ en todo $ p \ne p_0 $.

    Sea $  p \in S,\ p \ne p_0 $

    $$ df_{p}(v) = \dfrac{d}{dt}|_{t=0} \underbrace{f(\alpha (t))}_{|\alpha (t)-p| = <\alpha (t)-p_0,\alpha (t)-p_0>^{\frac{1}{2}}} = \dfrac{1}{2} <\alpha (t)-p_0,\alpha (t)-p_0>^{-\dfrac{1}{2}} 2 <\alpha'(t),\alpha(t)-p_0>|_{t=0} $$
    $$ = \dfrac{<v,p-p_0>}{|p-p_0|} $$
\end{exercise}

\begin{exercise}

    \textbf{i)}

    Es un difeomorfismo ya que es diferenciable, biyectiva y la inversa es diferenciable (es ella misma).

    $$ dA_{p}(v) = -v \hspace{5mm} \text{(lo vimos en un ejemplo)}  $$

    Como comentario, se puede ver que las diferenciales de las funciones lineales son ellas mismas.

    \textbf{ii)}

    $$ C = \{x^2+y^2=1\}\longrightarrow_{F} H = \{x^2+y^2-z^2 = 1\}\hspace{5mm}F(x,y,z) = (\sqrt{1+z^2}x,\sqrt{1+z^2}y,z)$$

    $$ \widetilde{F}: \mathbb{R} ^3 \to \mathbb{R} ^3\ \mathcal{C}^{\infty} \implies F|_{C} : C \to \mathbb{R} ^3\ \mathcal{C}^{\infty} \implies F\ \mathcal{C} ^{\infty}$$
    
    donde $ \widetilde{F}|_{C} = i \circ F $, con $ i $ la inclusión canónica. 

    $$ (\overline{x},\overline{y},\overline{z}) \in H \iff \overline{x} ^2 +\overline{y} ^2 -\overline{z} ^2  = (1+z^2)x^2 + (1+z^2)y^2-z^2 = (1+z^2)\underbrace{(x^2+y^2)}_{1}-z ^2  = 1 $$


    $$ F ^{-1}: H \to C\ F ^{-1}(u,v,w) = (\dfrac{u}{\sqrt{1+w^2}},\dfrac{v}{\sqrt{1+w^2}},w) $$

    Con lo que la inversa es también diferenciable.

    \textbf{iii)}

    Vemos primero la expresión de $ F $:
    $$ p = (x,y,z) \in \mathbb{S}^2 \setminus \{N,S\} \to F(p) \in H $$
    
    La recta con la que definimos $ F $ es $ r(t) = (0,0,z) + t(z,y,0) = (tx,ty,z) \in H $ para $ t > 0 $.
    $$ 1 = (tx)^2+(ty)^2-z^2 = t^2(x^2+y^2)-z^2\hspace{5mm}t^2=\dfrac{1+z^2}{x^2+y^2}= \dfrac{1+z^2}{1-z^2} $$
    Entonces, $ F(x,y,z) = \left(x \sqrt{\dfrac{1+z^2}{1-z^2}},y \sqrt{\dfrac{1+z^2}{1-z^2}},z\right) $

    Tomamos un abierto de $ W \subset \mathbb{R}^3 $ abierto donde exista $ \widetilde{F}$, entonces:
    $$ W = \{(x,y,z) \in \mathbb{R} ^3\ :\ -1<z<1 \} \equiv \mathbb{R}^2_{x}(-1,1)$$
    $$ \widetilde{F}(x,y,z) = \left(x \sqrt{\dfrac{1+z^2}{1-z^2}},y \sqrt{\dfrac{1+z^2}{1-z^2}},z\right)\ (x,y,z) \in W $$
    $$ \widetilde{F}\ \mathcal{C}^{\infty}\ \text{y}\ \widetilde{F}|_{\mathbb{S}^2 \setminus \{N,S\}} = F \implies  F\ es\ \mathcal{C}^{\infty}$$ 

    Tomamos $ H_1 = H \cap W = F(\mathbb{S}^2 \setminus \{N,S\}) $, $ H_1 \subset H $ es abierto $ \implies  H_1$ es una superficie.

    $$ G: \mathbb{S}^2 \setminus \{N,S\} \to H_1,\ G(x,y,z) = F(x,y,z) $$

    $$ G ^{-1}(x,y,z) = \left(x \sqrt{\dfrac{1-z^2}{1+z^2}},y \sqrt{\dfrac{1-z^2}{1+z^2}},z\right) \implies \text{ $ G $ es un difeomorfismo} $$
 

\end{exercise}


\begin{exercise}
    El argumento para ver que $ F $ es diferenciable seguiremos el mismo procedimiento de los dos ejercicios anteriores, obtendremos que es diferenciable en todo punto menos en $ p_0 $.
    $$ \widetilde{F}: \mathbb{R}^3 \setminus \{p_0\} \to \mathbb{R}^3 \ \widetilde{F}(p) = \widetilde{F}(x,y,z) = \dfrac{p-p_0}{|p-p_0|}\ \widetilde{F}|_{S}\ es\ \mathcal{C}^{\infty}$$

    $$ dFp(v) = (F \circ \alpha)' (0) = \dfrac{d}{dt}|_{t=0} |\alpha(t) - p_0 | ^{-1} (\alpha (t)-p_0) =$$$$= -\dfrac{1}{2}<\alpha(t)-p_0,\alpha(t)-p_0> ^{-\frac{3}{2}} 2<\alpha'(t),\alpha(t)-p_0>(\alpha(t)-p_0) + \dfrac{1}{|\alpha(t)-p_0|}\alpha'(t)|_{t=0} = $$
    $$ \dfrac{v}{|p-p_0|}- \dfrac{<v,p-p_0>}{|p-p_0|^3}(p-p_0)  $$

    Veamos la caracterización del núcleo:

    $$ \dfrac{v}{|p-p_0|}- \dfrac{<v,p-p_0>}{|p-p_0|^3}(p-p_0) = 0 \iff v = \lambda(p-p_0)  $$

    $$ \impliedby\ \text{Directo} $$
    $$ \implies  $$
    $$ v = \dfrac{<c,p-p_0>}{|p-p_0|^2}(p-p_0) $$

    Con lo que $ Ker(dFp) = <p-p_0> $
    


\end{exercise}

\begin{exercise}
    $$ F(x,y,z) = \left( \dfrac{x}{\sqrt{1-z^2}},\dfrac{y}{\sqrt{1-z^2}},z \right) $$
    \textbf{i)}

    Sea $  (x,y,z) \in \mathbb{S}^2 \setminus \{N,S\} $, veamos que $ F(x,y,z) \in C $:
    $$ \dfrac{x^2}{1-z^2}+\dfrac{y^2}{1-z^2} = \dfrac{x^2+y^2}{1-z^2} = \dfrac{1-z^2}{1-z^2} = 1 $$    

    Para comprobar que $ F $ es diferenciable, tomamos $ f: W \to \mathbb{R}^2\ W = \{(x,y,z) \in \mathbb{R}^3\ -1<z<1\}$ abierto. Como $F =  f|_{\mathbb{S}^2 \setminus \{N,S\}} = f \circ i $ y $ f $ es diferenciable, $ F $ es diferenciable.

    \textbf{ii)}

    Tomamos $ p \in \mathbb{S}^2 \setminus \{N,S\},\ v \in TpS  $ y $ \alpha: I \to \mathbb{S}^2 \setminus \{N,S\}\ \alpha(0) = p,\ \alpha'(0) = v $.

    $$ dFp(v) = \dfrac{d}{dt}|_{t=0}(F\circ \alpha)(t) = \dfrac{d}{dt}|_{t=0} \left( \dfrac{x(t)}{\sqrt{1-z^2(t)}},\dfrac{y(t)}{\sqrt{1-z^2(t)}},z(t) \right)=$$
     $$= \left( \dfrac{x'(t)(1-z^2(t)+x(t)z'(t)z(t))}{(1-z^2(t))^{1/2}}, \dfrac{y'(t)(1-z^2(t)+y(t)z'(t)z(t))}{(1-z^2(t))^{\frac{1}{2}}}, z'(t)\right) $$=$$ \left( \dfrac{v_1(1-z^2)+zv_3}{(1-z^2)^{\frac{2}{3}}},\dfrac{v_2(1-z^2)+zv_3}{(1-z^2)^{\frac{2}{3}}},v_3 \right)$$



\end{exercise}


\begin{exercise}

    Hay una errata en el enunciado, necesitamos la hipótesis $ <p,e_3> = 0 $ para el segundo apartado.

    \textbf{i)}

    Sea $ A = \mathbb{R}^3 \setminus \{(0,0,z)\} $ si $ F:A->\mathbb{R} $ tal que $ F(p) = \dfrac{1}{|p \times (0,0,1|} $ y $ F $ diferenciable, con lo que $ f $ es diferenciable ya que $ f = F|_{S} $.

    $$ dfp(v) = (f \circ \alpha) ' (0) = \dfrac{d}{dt}|_{t=0} <\dfrac{\alpha} (t) \wedge (0,0,1),\alpha(t) \wedge (0,0,1)> ^{\frac{1}{2}} =$$
    $$ = -\dfrac{1}{2} 2 <\alpha (t) \wedge (0,0,1),\alpha(t) (0,0,1)>^{\dfrac{3}{2}} <\alpha'(t) \wedge (0,0,1),\alpha (t) \wedge (0,0,1)> $$$$= \dfrac{-<\alpha'(t) \wedge (0,0,1),\alpha(t) \wedge (0,0,1)>}{|\alpha(t) \wedge (0,0,1)|^3}|_{t=0} = \dfrac{-<v \wedge (0,0,1),p \wedge (0,0,1)>}{|p \wedge (0,0,1)|^3} $$

    \textbf{ii)}

    Probaremos que $ <p,v> = 0 $

    $$ dfp = 0 \iff <v \wedge (0,0,1), p \wedge (0,0,1)> = 0 $$

    Si $ p = (x,y,z), v = (a,b,c) $, entonces la expresión anterior es:

    $$ <(a,b,c) \wedge (0,0,1),(x,y,z) \wedge (0,0,1)> = <(b,a,,0),(y,x,0)> = by+ax = 0 $$

    Volviendo a lo que queríamos probar:

    $$ <p,v> = 0 \iff xa+by+zc = zc = 0 $$
    que es cierto ya que $ <p,e_3> = 0 $

\end{exercise}

\begin{exercise}
    
    \textbf{i)}

    $ L = <n> $ $ n $ vector unitario base de $ L $, $ H_{\lambda} = \{p = (x,y,z) \in \mathbb{R} ^3: <p,n> = \lambda\} $
    
    $ S \subset H_{\lambda} $ para algún $ \lambda  \iff <p,n> cte\ Sea\ f: S \to \mathbb{R}\ f(p) = <p,n>$

    
    Para comprobar que es constante, calculamos su diferencial.
    $$ df_{p}(v) = <v,n> =_{n || N(p)} = 0 $$

    \textbf{ii)}

    Comprobaremos que $ f(p) = |p|^2 - <p,e_3>^2 $ es constante en la superficie.

    $$ df_{p}(v) = 2(<p,v>-<p,e_{3}><v,e_3>) = 2(<p,v> - <<p,e_3>e_3,v>) = $$
    $$ =2<\underbrace{p-<p,e_3>e_3}_{(x,y,0)||N(p)\footnote{$ p = <p,e_1>e_1 + <p,e_2>e_2 + <p,e_3>e_3 $}},v> = 0 $$

    \textbf{iii)}

    De nuevo comprobamos que $ f(p) = |p-p_0|^2 $ es constante con la diferencial:

    $$ df_{p}(v) = 2<\underbrace{p-p_0}_{||N(p)},v> = 0 $$

\end{exercise}

\begin{exercise}
    $$ X(u,v) = (\sin(u) \cos(v), \sin(u) \sin(v), \cos(u)) $$

    Usaremos las familias de curvas coordenadas obtenidas al fijar uno de los dos parámetros del mapa $ X $. Si fijamos $ u=u_0 $ obtenemos un paralelo. En el otro caso, si $ v = v_0 $, obtendremos un meridiano. Formamos así $ X_{u}(u,v_0) $

    Si $ \beta $ es el ángulo que forma la loxodroma tenemos que:

    $$ \cos(\beta) = \dfrac{<\alpha '(t),X_{u}(\widetilde{\alpha} (t))>}{|\alpha ' (t)||X_{u}(\widetilde{\alpha} (t)|} $$

    Definiendo $ \alpha (t) = X(u(t),v(t)) $ obtenemos:

    $$ \alpha'(t) = u'(t)X_{u}(\widetilde{\alpha}(t))X_{u}( \widetilde{\alpha (t)}) + v'(t)X_{v}(\widetilde{\alpha}(t)) $$

    Entonces:

    $$ \cos(\beta) = \dfrac{u'E+v'F}{|\alpha ' (t)|\sqrt{E}} $$

    Calculamos $ E,F,G $:
    $$ X_{u} = (\cos(u) \cos(v), \cos(u) \sin(v),-\sin(u)) \implies E = 1 $$
    $$ X_{u} = (-\sin(u) \sin(v), \sin(u) \cos(v),0) \implies F = 0 $$
    $$ G = \sin^2(u) $$

    Volviendo a la expresión anterior:

    $$ \cos(\beta) = \dfrac{u'}{\sqrt{u'^2+v'^2+\sin^2(u)}} $$

    $$ \cos^2(\beta) (u'^2+v'^2 \sin^2(u))=u'^2  $$
    $$ v'^2 \sin^2(u) = \left( \dfrac{1}{\cos^2(\beta)} -1\right)u'^2 = \tan^2(\beta )u'^2 $$
    $$ v' = \tan(\beta) \dfrac{u'}{\sin(u)} \implies_{integrando} v+cte = \pm \log{\tan\left(\dfrac{u}{2}\right)} \tan(\beta ) $$

    Con lo que $ \alpha (t) = X\left(t,\pm\log{\dfrac{t}{2}}\tan(\beta )+cte\right) $

\end{exercise}

\begin{exercise}
    $$ X(u,v) = (\cos(u)\cosh(v), \sin(u),\cosh(v),v) $$
    $$ X_{u}(u,v) = (-\sin((u)\cosh(v), \cos(u)\cosh(v),0) $$
    $$ X_{v}(u,v) = (\cos((u)\sinh(v), \sin(u)\sinh(v),1) $$
    $$ E = \cosh^2(v)\hspace{5mm}F = 0 \hspace{5mm}G = \cosh ^2 (v) $$

    Aplicaremos la misma estrategia que usamos para el toro, calcularemos el área para una serie de regiones que aproximan al catenoide:

    $$ A(R_{\varepsilon}) = \int\limits_{-1}^{1}\int\limits_{\varepsilon}^{2 \pi - \varepsilon} \cosh ^2(v)dudv= (2 \pi-2 \varepsilon) \left( 1 + \dfrac{\sinh(2)}{2} \right)$$
    
    Y solo hay que tomar límites para $ \varepsilon \to 0 $
\end{exercise}

\begin{exercise}

    \textbf{i)}

    $$ X_{u}(u,v) = \left(\cos(u)\cos(v), \cos(u)\sin(v),-\sin(u)+\dfrac{1}{\sin(u)}\right) = $$
    $$=  \left(\cos(u)\cos(v),\cos(u) \sin(v), \dfrac{\cos^2(u)}{\sin(v)}\right) $$
    $$ E = \dfrac{\cos^2(u)}{\sin^2(u)} = \cot^2(u) $$
    $$ F = 0 $$
    $$ G = \sin^2(u)\sin^2(v) + \sin^2(u)\cos^2(v)= \sin^2(u) $$

    \textbf{ii)}

    Tomamos la región $ R_{\varepsilon} = X((\varepsilon,\dfrac{\pi}{2}-\varepsilon)\times [\varepsilon,2\pi-\varepsilon]) $

    $$ A(R_{\varepsilon}) = \int\limits_{\varepsilon}^{{\pi}/{2}-\varepsilon} \int\limits_{\varepsilon}^{2 \pi-\varepsilon} \sqrt{EG-F^2}dudv = \int\limits_{\varepsilon}^{\dfrac{\pi}{2}-\varepsilon} \cos(u)du (2\pi-2 \varepsilon) = 2 (\sin(\dfrac{\pi}{2}-\varepsilon)-\sin(\varepsilon)) (\pi-\varepsilon) $$

    $$ A(R) = \lim_{\varepsilon \to 0} A(R_{\varepsilon}) = 2 \pi $$

\end{exercise}

\begin{exercise}
    
    \textbf{i)}

    $$ X_{u}(u,v) = \alpha' (s) + r(\cos(\theta)n'(s)+\sin(\theta)b'(s)) = $$
    $$_= {Frenet} t(s)+ r(\cos(\theta)(-k(s)t(s)-\tau(s)bs(s))+\sin(\theta)\tau(s)n(s)) = $$
    $$ = t(s)(1-r \cos(\theta)k(s))- r \cos(\theta)\tau(s)b(s) + r\sin(\theta)\tau(s)n(s) $$
    $$ X_{v}(u,v) = r(-\sin(\theta)n(s)+\cos(\theta)b(s)) $$

    Como $ \{t,n,b\} $ es una base ortonormal podemos calcular $ E =  <X_{s},X_{s}> $ como la suma de los coeficientes de los vectores de la base al cuadrado. Lo mismo ocurre para $ G $:

    $$ E = <X_{s},X_{s}> = (1-r \cos(\theta)k(s))^2+r^2(\tau(s))^2 $$
    $$ F = <X_{u},X_{\theta}> = ... = -r^2\tau(s)$$
    $$ G = <X_{\theta},X_{\theta}> = r^2(\sin^2(\theta)+\cos^2(\theta)) = r^2$$

    \textbf{ii)}

    $$ I = (a,b) \to I_{\varepsilon} = [a+\varepsilon,b-\varepsilon] \to R_{\varepsilon} = X(I_{\varepsilon} \times [\varepsilon,2 \pi- \varepsilon]) $$
    $$ A(R_{\varepsilon}) = \int\limits_{\varepsilon}^{2 \pi- \varepsilon} \int\limits_{I_{\varepsilon}}^{} \sqrt{EG-F^2}dsd \theta = r \iint (1- \cos(\theta)k(s)r)dsd \theta =  $$
    Tendremos que probar que $ r^2 \int\limits_{I_{\varepsilon}}^{}k(s)ds \int\limits_{\varepsilon}^{2\pi - \varepsilon}\cos(\theta) = 0$

    $$ EG-F^2 = ... = r^2(1-r \cos(\theta)k(s))^2 \implies \sqrt{EG-F^2} = r(1-k(s)r \cos(\theta))\footnote{Sabemos por el enunciado que no tenemos que poner valores absolutos al ser $ 1-rk(s)\cos(\theta)>0 $} $$

\end{exercise}


\end{document}
