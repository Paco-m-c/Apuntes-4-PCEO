\documentclass[openany]{book}
\usepackage[utf8]{inputenc}
\usepackage{verbatim}
\usepackage[hypertexnames=false]{hyperref}
\usepackage{amstext}
\usepackage{array}
\newcolumntype{C}{>{$}c<{$}}


%%%%%%%%%%%%%%%%%%%%%%%

%%%%%%%%%%%%%%%%%%%%%%%
% HOLA PACO
% ESTE ES EL ARCHIVO DE LAS DEFINICIONES ESTRUCTURALES
% VERSION 1.1 NOMÁS
%
% AUTOR ORIGINAL:
% EDUARDO (CHITO) BELMONTE GUILLAMÓN
%
% ESTE ARCHIVO ES COMUNISTA, PUEDES COMPARTIRLO SI QUIERES
%%%%%%%%%%%%%%%%%%%%%%%

%----------------------------------
%     PAQUETICOS QUE SE USAN
%----------------------------------

%--------------------------
%    PARA USAR INKSCAPE
%---------------------------
\usepackage{import}
\usepackage{hyperref}
\usepackage{xifthen}
\usepackage{pdfpages}
\usepackage{transparent}

\newcommand{\incfig}[1]{%
    \def\svgwidth{\columnwidth}
    \import{./figures/}{#1.pdf_tex}
}

\newcommand{\custincfig}[2]{%
    \def\svgwidth{#1}
    \import{./figures/}{#2.pdf_tex}
}
\newcommand{\textnexttofig}[3]{
  \begin{minipage}[l]{0.45\textwidth}
    \custincfig{#1}{#2}
  \end{minipage}
  \begin{minipage}[l]{0.45\textwidth}
    #3
  \end{minipage}
}

%%%%%%%%% FIN DEL INKSCAPE

\usepackage{parskip} % Pa parrafos wapos
\setlength{\parindent}{0.5cm} % Pa la sangría
\usepackage{graphicx} % Pa meter las imágenes
\graphicspath{{Images/}} % La ruta a las imágenes

\usepackage{tikz} % Pa dibujar cosichuelas guapas

\usepackage[spanish]{babel} % PA QUE ESTÉ EN ESPAÑOL NOMÁS

\usepackage{enumitem} % Para personalizar las LISTAS YEAH

\setlist{nolistsep} % Pa que las listas estén junticas

\usepackage{booktabs} % Esta sirve para hacer tablas fancy con multicolumns y tal pero no tengo ni puta idea de usarla

\usepackage{xcolor} % PA DEFINIR LOS COLORINES
\definecolor{turquoise}{RGB}{21,103,112} % Es un turquesica así formal
\definecolor{violet}{RGB}{ 110, 6, 187 } % Color maricón

%-------------------------------------------------
%     MÁRGENES
%-------------------------------------------------

\usepackage{geometry}
\geometry{
    top=3cm,
    bottom=3cm,
    left=3cm,
    right=3cm,
    headheight=14pt,
	footskip=1.4cm,
	headsep=10pt,
}

\usepackage{avant} % Esto es una fuente para encabezados

%\usepackage{mathptmx} % Usar simbolitos matemáticos chulos

\usepackage{microtype} % Para fuentes de maricones

\usepackage[utf8]{inputenc} % Pa los acentos

\usepackage[T1]{fontenc}

%-------------------------------------------------
% Bibliografía e índice
%-------------------------------------------------

\usepackage{makeidx} % Pa hacer un índice
\makeindex

\usepackage{titletoc}   % Para manipular la tabla de contenidos

\contentsmargin{0cm}    % Para eliminar el margen por defecto

\usepackage{titlesec} % Pa cambiar los titulos skere

\titleformat
{\chapter} % command
[display] % shape
{\centering\bfseries\Huge\normalfont} % format
{\color{turquoise}  {\normalsize\MakeUppercase{Capítulo} \thechapter }} % label
{-0.5cm} % sep
{
    \color{turquoise}
    \rule{\textwidth}{3pt}
    \vspace{1ex}
    \centering
    \setcounter{ex}{0}
    \setcounter{dummy}{0}
} % before-code
[
\vspace{-0.5cm}%
\rule{\textwidth}{3pt}
] % after-code


\titleformat{\part}
[display]
{\centering\bfseries\Huge\normalfont}
{\color{turquoise} {\normalsize \MakeUppercase{Asignatura}}}
{0pt}
{\color{turquoise}
\vspace{-0.6cm}
\rule{\textwidth}{3pt}
\vspace{1ex}
\setcounter{chapter}{0}
\setcounter{section}{0}
\setcounter{dummy}{0}
\centering
}


\titleformat{\section}
{\normalfont\Large\bfseries}{\color{turquoise}\thesection\ - }{0.5em}{}

\usepackage{fancyhdr}   % Necesario para el encabezado y el pie de página

\pagestyle{fancy}   %Para modificar los encabezados
\fancyhf{}          %Para eliminar los encabezados y pies de página por defecto.
\fancyhead[LE,RO]{\sffamily\normalsize\thepage}
\fancyfoot[C]{Ampliación de Probabilidad}
%HACER

\usepackage{amsmath,amsfonts,amssymb,amsthm,cancel} % PARA LAS MATES

%   LINEA 199, HACER CAPULLADAS

\newtheoremstyle{turquoisebox}
{0pt} %Espacio encima
{0pt} %Espacio abajo
{\normalfont} % Fuente del cuerpo
{} % Cantidad de identado
{\small\ssfamily\color{turquoise}} % Fuente en la que pone "TEOREMA"
{:} % Puntuación tras el teorema
{0.25em} %Espacio tras el teorema
{\thmname{#1}\thmnumber{#2}} %No sé si esto funciona


\newcounter{dummy}
\newcounter{ex}
\newtheorem{teoremote}[dummy]{\color{turquoise}Teorema}
\newtheorem{propositiont}{\color{turquoise}Proposición}[section]
\newtheorem{lemmat}{\color{turquoise}Lema}[section]
\newtheorem{definitionT}{\color{turquoise}Definición}[section]
\newtheorem{exerciseT}[ex]{Ejercicio}
\newtheorem{examplote}[ex]{\color{turquoise}Ejemplo}
\newtheorem{methodT}[dummy]{\color{turquoise}Método}


\RequirePackage[framemethod=default]{mdframed} % Required for creating the theorem, definition, exercise and corollary boxes

%Caja de teoremas

\newmdenv[skipabove=7pt,
skipbelow=7pt,
backgroundcolor=black!5,
linecolor=turquoise,
innerleftmargin=5pt,
innerrightmargin=5pt,
innertopmargin=5pt,
leftmargin=0cm,
rightmargin=0cm,
linewidth=3pt,
innerbottommargin=5pt]{tBox}

\newmdenv[skipabove=7pt,
skipbelow=7pt,
backgroundcolor=black!5,
linecolor=turquoise,
innerleftmargin=5pt,
innerrightmargin=5pt,
innertopmargin=5pt,
leftmargin=0cm,
rightmargin=0cm,
linewidth=1pt,
innerbottommargin=5pt]{pBox}

\newmdenv[skipabove=7pt,
skipbelow=7pt,
backgroundcolor=violet!7,
linecolor=turquoise,
innerleftmargin=5pt,
innerrightmargin=5pt,
innertopmargin=5pt,
leftmargin=0cm,
rightmargin=0cm,
rightline=false,
topline=false,
bottomline=false,
linewidth=4pt,
innerbottommargin=5pt]{mBox}

\newmdenv[skipabove=7pt,
skipbelow=7pt,
rightline=false,
leftline=true,
topline=false,
bottomline=false,
linecolor=turquoise,
innerleftmargin=5pt,
innerrightmargin=5pt,
innertopmargin=0pt,
leftmargin=0cm,
rightmargin=0cm,
linewidth=4pt,
innerbottommargin=0pt]{dBox}

\newmdenv[skipabove=7pt,
skipbelow=7pt,
rightline=false,
leftline=true,
topline=false,
bottomline=false,
backgroundcolor=black!3,
linecolor=turquoise!50,
innerleftmargin=5pt,
innerrightmargin=5pt,
innertopmargin=0pt,
innerbottommargin=5pt,
leftmargin=0cm,
rightmargin=0cm,
linewidth=4pt]{eBox}

\newmdenv[skipabove=7pt,
skipbelow=7pt,
leftline=true,
topline=false,
rightline=false,
bottomline=false,
backgroundcolor=cyan!5,
linecolor=turquoise,
innerleftmargin=5pt,
innerrightmargin=5pt,
innertopmargin=0pt,
innerbottommargin=5pt,
leftmargin=0cm,
rightmargin=0cm,
linewidth=4pt]{exBox}

\newenvironment{theorem}{\begin{tBox}\begin{teoremote}}{\end{teoremote}\end{tBox}}
\newenvironment{proposition}{\begin{pBox}\begin{propositiont}}{\end{propositiont}\end{pBox}}
\newenvironment{lemma}{\begin{pBox}\begin{lemmat}}{\end{lemmat}\end{pBox}}
\newenvironment{method}{\begin{mBox}\begin{methodT}}{\end{methodT}\end{mBox}}
\newenvironment{definition}{\begin{dBox}\begin{definitionT}}{\end{definitionT}\end{dBox}}
\newenvironment{exercise}{\begin{eBox}\begin{exerciseT}}{\hfill{\color{black}}\end{exerciseT}\end{eBox}}
\newenvironment{example}{\begin{exBox}\begin{examplote}}{\end{examplote}\end{exBox}}
\newenvironment{demonstration}{\begin{flushright}
      \color{turquoise} \textbf{Demostración}
\end{flushright}
}{\begin{flushright}
  $\square$
\end{flushright}}

\usepackage{geometry}
\geometry{
    top=3cm,
    bottom=3cm,
    left=3cm,
    right=3cm,
    headheight=14pt,
    footskip=1.4cm,
    headsep=10pt,
}
\usepackage{graphicx}
\title{Demostraciones de Geometría Global de Superficies}
\author{Paco Mora y Chito Belmonte\\Apuntes PaChito™}
\date{\today\\ \vspace{1cm}
\textbf{Pequeño inciso: Estos apuntes apoyan algunas de sus demostraciones en el libro \textit{Un curso de Geometría Diferencial}, con lo cual puede que algunas demostraciones no coincidan con lo visto en clase.}}

\begin{document}


\maketitle
\tableofcontents

\chapter{Capítulo 1}

\begin{center}
\textbf{Proposición 1.4, la derivada covariante es intrínseca.}
\end{center}

\begin{demonstration}
  Desde luego, $D V / d t \in \mathfrak{X}(\alpha)$. Además, para una curva fija $\alpha$, la derivada covariante puede verse como un operador, $D / d t$, de la forma
  $$
  \begin{array}{lllllll}
    \dfrac{D}{dt} : &  \mathfrak{X} & \to & \mathfrak{X} &&&\\
    & V & \sim & \dfrac{DV}{dt}: & I & \to & \mathbb{R}^{ 3 } \\
    &&&&t & \sim & V'(t)<V'(t), N(t)>N(t)
  \end{array}
  $$
Este operador es independiente de la orientación elegida para la superficie, pues estamos tomando sólo la parte tangente de $V^{\prime}$ (obsérvese que si cambiamos $N$ por $-N$ en la fórmula, el resultado es el mismo).

Por otro lado, sólo depende de los coeficientes de la primera forma fundamental, afirmación que vamos a probar a continuación.

Para ello, sea $(U, X)$ una parametrización de la superficie $S$ y, como viene siendo habitual, $\alpha(t)=X(u(t), v(t))$. Si $V \in \mathfrak{X}(\alpha)$, entonces $V(t) \in T_{\alpha(t)} S$, por lo que puede expresarse de la forma
$$
V(t)=a(t) X_{u}(u(t), v(t))+b(t) X_{v}(u(t), v(t)) .
$$
Ahora, calculamos $V^{\prime}(t)$, utilizando las fórmulas de Gauss para expresar $X_{u u}$, $X_{u v}$ y $X_{v v}$ en términos de la base $\left\{X_{u}, X_{v}, N\right\}$ :
$$
\begin{aligned}
V^{\prime}=& a^{\prime} X_{u}+a\left(X_{u u} u^{\prime}+X_{u v} v^{\prime}\right)+b^{\prime} X_{v}+b\left(X_{v u} u^{\prime}+X_{v v} v^{\prime}\right) \\
=& a^{\prime} X_{u}+a\left[\left(\Gamma_{11}^{1} X_{u}+\Gamma_{11}^{2} X_{v}+e N\right) u^{\prime}+\left(\Gamma_{12}^{1} X_{u}+\Gamma_{12}^{2} X_{v}+f N\right) v^{\prime}\right] \\
&+b^{\prime} X_{v}+b\left[\left(\Gamma_{12}^{1} X_{u}+\Gamma_{12}^{2} X_{v}+f N\right) u^{\prime}+\left(\Gamma_{22}^{1} X_{u}+\Gamma_{22}^{2} X_{v}+g N\right) v^{\prime}\right] \\
=&\left(a^{\prime}+a u^{\prime} \Gamma_{11}^{1}+a v^{\prime} \Gamma_{12}^{1}+b u^{\prime} \Gamma_{12}^{1}+b v^{\prime} \Gamma_{22}^{1}\right) X_{u} \\
&+\left(b^{\prime}+a u^{\prime} \Gamma_{11}^{2}+a v^{\prime} \Gamma_{12}^{2}+b u^{\prime} \Gamma_{12}^{2}+b v^{\prime} \Gamma_{22}^{2}\right) X_{v}+\left(a e u^{\prime}+a f v^{\prime}+b f u^{\prime}+b g v^{\prime}\right) N .
\end{aligned}
$$


En consecuencia, la derivada covariante $D V / d t=\left(V^{\prime}\right)^{\top}$ se escribe como
$$
\begin{aligned}
\frac{D V}{d t}=&\left(a^{\prime}+a u^{\prime} \Gamma_{11}^{1}+\left(a v^{\prime}+b u^{\prime}\right) \Gamma_{12}^{1}+b v^{\prime} \Gamma_{22}^{1}\right) X_{u} \\
&+\left(b^{\prime}+a u^{\prime} \Gamma_{11}^{2}+\left(a v^{\prime}+b u^{\prime}\right) \Gamma_{12}^{2}+b v^{\prime} \Gamma_{22}^{2}\right) X_{v}
\end{aligned}
$$
Obsérvese que $D V / d t$ depende sólo de los símbolos de Christoffel y, por tanto, de la primera forma fundamental exclusivamente. En otras palabras, la derivada covariante es algo intrínseco; sus propiedades permanecen invariantes por isometrías.
\end{demonstration}

\begin{center}
\textbf{Proposición 1.5, linealidad y regla de Leibniz para la derivada covariante.}
\end{center}
\begin{demonstration}
  Las tres propiedades se demuestran trivialmente \footnote{Tu puta madre por si acaso.} :
    $$ I $$
 $\frac{D}{d t}(V+W)=\left[(V+W)^{\prime}\right]^{\top}=\left(V^{\prime}+W^{\prime}\right)^{\top}=\left(V^{\prime}\right)^{\top}+\left(W^{\prime}\right)^{\top}=\frac{D V}{d t}+\frac{D W}{d t}$.
  $$ II $$
 $\frac{D}{d t}(f V)=\left[(f V)^{\prime}\right]^{\top}=\left(f^{\prime} V+f V^{\prime}\right)^{\top}=\left(f^{\prime} V\right)^{\top}+\left(f V^{\prime}\right)^{\top}=f^{\prime} V+f \frac{D V}{d t}$,
pues $V^{\top}=V$, ya que $V \in \mathcal{X}(\alpha)$.

$$ III $$
La propiedad se prueba de forma similar: como $V, W \in \mathfrak{X}(\alpha)$, entonces $\left\langle\left(V^{\prime}\right)^{\perp}, W\right\rangle=\left\langle V,\left(W^{\prime}\right)^{\perp}\right\rangle=0$, de donde
$$
\begin{aligned}
\langle V, W\rangle^{\prime} &=\left\langle V^{\prime}, W\right\rangle+\left\langle V, W^{\prime}\right\rangle=\left\langle\left(V^{\prime}\right)^{\top}+\left(V^{\prime}\right)^{\perp}, W\right\rangle+\left\langle V,\left(W^{\prime}\right)^{\top}+\left(W^{\prime}\right)^{\perp}\right\rangle \\
&=\left\langle\frac{D V}{d t}, W\right\rangle+\left\langle V, \frac{D W}{d t}\right\rangle
\end{aligned}
$$
\end{demonstration}

\begin{center}
\textbf{Proposición 1.7}
\end{center}
\begin{demonstration}
  La parte $I$ es trivial \footnote{Cada vez que lea 'es trivial' voy a meter un 'tu puta madre por si acaso'.}
    $$ II $$
  Al ser $D V / d t=\mathbf{0}$, sabemos que $V^{\prime}(t)$ está en la dirección del normal a la superficie, y análogamente $W^{\prime}(t)$. En consecuencia, como $V, W \in \mathfrak{X}(\alpha)$, se tiene que $\left\langle V, W^{\prime}\right\rangle=\left\langle V^{\prime}, W\right\rangle=0$ y, por tanto, $\langle V, W\rangle^{\prime}=\left\langle V^{\prime}, W\right\rangle+\left\langle V, W^{\prime}\right\rangle=0$.

  Esto demuestra que $\langle V, W\rangle$ es constante.
\end{demonstration}

\begin{center}
\textbf{Proposición 1.8, la ecuación diferencial extrínseca de los campos paralelos}
\end{center}
\begin{demonstration}
  PANIC
\end{demonstration}

\begin{center}
\textbf{Proposición 1.9, la ecuación diferencial intrínseca de los campos paralelos}
\end{center}
\begin{demonstration}
  PANIC
\end{demonstration}
\newpage
\begin{center}
\textbf{Teorema 1.10, existencia y unicidad de campos paralelos}\footnote{Hay una demostración extrínseca y otra intrínseca en el libro. He decidido copiar la intrínseca porque leyéndola en diagonal he determinado que sería más sencilla, si queréis buscar la otra (fruto de vuestra curiosidad o esquizofrenia paranoide), está en el libro.}
\end{center}
\begin{demonstration}

  Hay que encontrar $V \in \mathfrak{X}(\alpha)$ tal que $D V / d t=\mathbf{0}$.

  Sea $(U, X)$ una parametrización de $S$ con $\alpha\left(t_{0}\right) \in X(U)$, y sea $\alpha(t)=X(u(t), v(t))$. Utilizando la expresión de la demostración de la proposición 1.4 para la derivada covariante de un campo $V \in \mathfrak{X}(\alpha)$, como $D V / d t=\mathbf{0}$, los coeficientes de los vectores $X_{u}$ y $X_{v}$ tienen que anularse; luego si representamos $V$ como $V(t)=a(t) X_{u}(u(t), v(t))+b(t) X_{v}(u(t), v(t))$, se verificará que
$$
\left\{\begin{array}{l}
a^{\prime}+a u^{\prime} \Gamma_{11}^{1}+\left(a v^{\prime}+b u^{\prime}\right) \Gamma_{12}^{1}+b v^{\prime} \Gamma_{22}^{1}=0 \\
b^{\prime}+a u^{\prime} \Gamma_{11}^{2}+\left(a v^{\prime}+b u^{\prime}\right) \Gamma_{12}^{2}+b v^{\prime} \Gamma_{22}^{2}=0
\end{array}\right.
$$

En otras palabras, se prueba que siempre es posible 'fijar un rumbo' en la superficie. Una vez fijado, ya se puede hablar de si se mantiene o no constante la dirección de cualquier otro campo.

Tenemos por tanto un sistema de ecuaciones diferenciales cuya condición inicial es $$V\left(t_{0}\right)=V_{0}=\left(a\left(t_{0}\right), b\left(t_{0}\right)\right)$$ El teorema de existencia y unicidad de soluciones para tales sistemas establece el resultado buscado (además, se puede obtener la solución de forma explícita al resolverlo, siempre y cuando esto sea posible).

\end{demonstration}








\chapter{Capítulo 2}

\begin{proposition}
  { \color{turquoise} \textbf{Proposición exclusiva de \textit{Apuntes PaChito ™}}}

  Sea $\alpha : I \to S \subseteq \mathbb{R}^{ 3 } $ una parametrización. Son equivalentes:

  \begin{enumerate}
    \item $\alpha$ es pregeodésica
    \item $\exists \beta (u) = \alpha (h(u))$ una reparametrización de $\alpha /\beta $ es geodésica.
    \item La reparametrización parametrizada por el arco de $\alpha $ es geodésica.
  \end{enumerate}
\end{proposition}
\begin{demonstration}
  $$ 1 \implies 2 $$

  Directo por la definición

  $$ 3 \implies 1 $$

  Por la definición directo

  $$ 2 \implies 3 $$

  ¿Te imaginas que es directo por la definición? Pues no. Mala suerte.

  Tomamos una reparametrización de $I$:

  $$ I \to^\alpha S $$
  $$ J \to^{h(u)} I $$
  $$ J \to^{\beta (u) = \alpha (h(u)) \text{ que es geodésica}} S $$

  Esto es un triangulito si lo dibujáis... Estaría bien que yo también lo hiciera pero soy un \textit{vago}.

  $$ ||\beta '(u)||= c = h'(u) \cdot ||\alpha '(h(u))|| $$

  Si $c=1 \to \beta (u)$ es la reparametrizavión por arco de $\alpha $

  Si $c \ne 1 \to $ Tomo $\underbrace{ \gamma (s) = \beta(\frac{u}{c}) }_{ GEODESICA } = \alpha (h(\frac{u}{c}))$
  $$ \gamma '(s) = \dfrac{1}{c} \cdot  \beta'\left(\frac{s}{c}\right) $$
\end{demonstration}











\chapter{Capítulo 3}

\begin{center}
\textbf{Lema 3.2, Lema de homogeneidad de las geodésicas}
\end{center}
  Lo más importante del lema son las fórmulas que aparecen centradas en su enunciado.
\begin{demonstration}
  Sea $p \in S$ y sea un vector tangente $v \in T_pS$. Entonces, $\exists \gamma _v \ : \ I_v \to S$. Como $\lambda \ne 0$, entonces $ \lambda v \in T_pS $.

  Defino ahora $ \alpha (t) = \gamma _v (\lambda t) $.
  $$ I_ \alpha = \{ t \in \mathbb{R} \ : \ \lambda t \in I_v  \} $$
  $$ t \in I_ \alpha \iff \lambda t \in I_v  \iff t \in \dfrac{1}{\lambda } \cdot I_v $$

  Luego otenemos que $\dfrac{1}{\lambda} I_v = I_ \alpha $. De esto deducimos que $\alpha (0) = p = \gamma _v(0)$.

  $$ \alpha '(t) = \lambda \gamma '_v( \lambda t) \implies \alpha '(0) = \lambda \gamma' _v (0() = \lambda v$$

  Y se obtiene que $\alpha$ es una geodésica, por ser reparametrización afín de una geométrica.

  $$ \left\{ \begin{array}{l}
    \alpha (t) = \gamma _v (\alpha t)\\
    I_ \alpha  = \dfrac{1}{\lambda} I_v
  \end{array} \right. \in  \mathcal{J}_{p, \lambda v} = \{ (I_{\alpha , \alpha }), ... \} \implies \dfrac{1}{\lambda} I_v \subseteq I_ {\alpha v}$$
  $$\forall t \in \dfrac{1}{\lambda} I_v, \ \gamma _{\lambda v}(t) = \lambda _v( \lambda t)$$

  Llamo ahora $w = \lambda v$.
  $$ \mu = \dfrac{1}{\lambda} \implies \mu w = v $$
  $$ \dfrac{1}{\mu} \cdot I_w \subseteq I_{\mu w}\ y\ \forall t \in \underbrace{ \dfrac{1}{\mu} I_w }_{ \lambda I_{\lambda v} } $$
  $$ \gamma _{\mu w}(t) = \gamma _w(\mu t) \hspace{1cm} \dfrac{1}{\mu} \cdot I_w = \lambda I_ \lambda v \subseteq I_v$$
  $$ I_{\lambda v} \subseteq \dfrac{1}{\lambda} I_v $$

  Al final he tenido que copiar tal cual lo que pone en la pizarra y no tengo ni idea de si he demostrado algo o no.
\end{demonstration}

\begin{center}
\textbf{Teorema 3.3, Propiedades de la aplicación exponencial}
\end{center}

Primero tenemos que demostrar esta propiedad
$$ p \in S, \ v \in T_pS \implies  iv \in \mathcal{D} _p, \ \forall t \in I_v \supset (- \varepsilon , \varepsilon ) $$
\begin{demonstration}
  $$ \left\{ \begin{array}{l}
    t=0 \to OK\\
    t \ne 0, \hspace{1cm} tv \in \mathcal{D}_p \iff 1 \in I_{tv} = \dfrac{1}{t} \cdot I_v\ si\ 1=\dfrac{1}{t}t
  \end{array} \right. $$
  $$ exp_p(tv) = \gamma _{tv}(1) = \gamma _v (t) , \ \forall t \in I_v $$
\end{demonstration}
Fin de la demostración inciso, comenzamos a demostrar el teorema.
\begin{demonstration}
  $$ I $$
  Que sea estrellado nos dice que el segmento que une cualquier punto con el origen no se sale del conjunto. En otras palabras,
  $$ \forall \underbrace{ v \in \mathcal{D}_p }_{ 1 \in I_v \implies [0,1] \subset I \ \forall t \in [0,1] }, \ [0,v] = \{ tv, \ 0 \leq t \leq 1 \} \subset \mathcal{D}_p $$

  Del inciso se deduce que $t \in I_v \implies tv \in \mathcal{D}_p$. Notamos lo siguiente:
  $$ \forall v \in T_pS, \ \exists (- \varepsilon , \varepsilon ) \subset I_v $$
  Y justo debajo ha escrito
  $$ \exists \varepsilon > 0 \ : \ \forall |t| < \varepsilon , \ tv \in \mathcal{D}_p $$
  $$ II $$
  Este dice el sensei Alías que nos lo creamos y yo \textbf{me lo creo.}
  $$ III $$
  $$ \exists \underbrace{ U }_{ \text{entorno de }0 }\footnote{Para el sensei Alías, decir entorno implica que es abierto} \in \mathcal{D}_p \ : \ exp_{p|\mathcal{U}} \ : \  \underbrace{ \mathcal{U} }_{ \subset T_pS } \to \underbrace{ V }_{ \subset S } \text{ es difeomorfismo}$$

  Para comprobar si es un difeomorfismo, tenemos que ver que la diferencial es un isomorfismo lineal. Vamos a ver si esto es así:

  $$ \forall v \in \mathcal{D}_p \to d(exp_p)_v \ : \ \underbrace{ T_v(\mathcal{D}_p) }_{ T_v(T_pS) } \to T_{exp_p(v)}S $$

  $$ \forall w  \in T_pS \hspace{1cm} \alpha (t) = v + tw $$
  $$ \alpha \ : \ I \to \mathcal{D}_p \subset T_pS $$
  $$ \alpha (0) = v \hspace{1cm} \alpha '(0) = w \hspace{1cm} \to \hspace{1cm}w \in T_v(T_pS)$$

  Vamos al caso particular $v=0$.
  $$ d(exp_p)_0 \ : \ T_0(TpS) = T_pS \to T_{exp_p(v)}S = T_pS$$
  $$ \forall w \in T_pS = T_0(T_pS) $$
  $$ d(exp_p)_v(w) = \dfrac{d}{dt}_{t=0} exp_p(\alpha (t)) $$

  Tenemos que encontrar una curva que cumpla
  $$ \forall \alpha (t) \ : \ (- \varepsilon , \varepsilon ) \to T_pS $$
  $$ \left. \begin{array}{l}
    \alpha (0) = 0\\
    \alpha '(0) = w
  \end{array} \right\} \alpha (t) = t w $$

  $$ exp_p(\alpha (t)) = exp_p(tw) = \gamma _{tw}(1) = \gamma _w(t) $$
  $$ d(exp_p)_0(w) = \gamma '_w(0) = w \implies d(exp_p)_0 = Id $$
\end{demonstration}

\end{document}
