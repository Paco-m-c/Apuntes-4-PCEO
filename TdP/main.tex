\documentclass[openany]{book}
\usepackage[utf8]{inputenc}
\usepackage{verbatim}
\usepackage[hypertexnames=false]{hyperref}
\usepackage{amstext} 
\usepackage{array}
\usepackage{tcolorbox}
\usepackage{amssymb}
\usepackage{amsmath}
\usepackage{graphicx}
\newcolumntype{C}{>{$}c<{$}} 


%%%%%%%%%%%%%%%%%%%%%%%

%%%%%%%%%%%%%%%%%%%%%%%
% HOLA PACO
% ESTE ES EL ARCHIVO DE LAS DEFINICIONES ESTRUCTURALES
% VERSION 1.1 NOMÁS
%
% AUTOR ORIGINAL:
% EDUARDO (CHITO) BELMONTE GUILLAMÓN
%
% ESTE ARCHIVO ES COMUNISTA, PUEDES COMPARTIRLO SI QUIERES
%%%%%%%%%%%%%%%%%%%%%%%

%----------------------------------
%     PAQUETICOS QUE SE USAN
%----------------------------------

%--------------------------
%    PARA USAR INKSCAPE
%---------------------------
\usepackage{import}
\usepackage{hyperref}
\usepackage{xifthen}
\usepackage{pdfpages}
\usepackage{transparent}

\newcommand{\incfig}[1]{%
    \def\svgwidth{\columnwidth}
    \import{./figures/}{#1.pdf_tex}
}

\newcommand{\custincfig}[2]{%
    \def\svgwidth{#1}
    \import{./figures/}{#2.pdf_tex}
}
\newcommand{\textnexttofig}[3]{
  \begin{minipage}[l]{0.45\textwidth}
    \custincfig{#1}{#2}
  \end{minipage}
  \begin{minipage}[l]{0.45\textwidth}
    #3
  \end{minipage}
}

%%%%%%%%% FIN DEL INKSCAPE

\usepackage{parskip} % Pa parrafos wapos
\setlength{\parindent}{0.5cm} % Pa la sangría
\usepackage{graphicx} % Pa meter las imágenes
\graphicspath{{Images/}} % La ruta a las imágenes

\usepackage{tikz} % Pa dibujar cosichuelas guapas

\usepackage[spanish]{babel} % PA QUE ESTÉ EN ESPAÑOL NOMÁS

\usepackage{enumitem} % Para personalizar las LISTAS YEAH

\setlist{nolistsep} % Pa que las listas estén junticas

\usepackage{booktabs} % Esta sirve para hacer tablas fancy con multicolumns y tal pero no tengo ni puta idea de usarla

\usepackage{xcolor} % PA DEFINIR LOS COLORINES
\definecolor{turquoise}{RGB}{21,103,112} % Es un turquesica así formal
\definecolor{violet}{RGB}{ 110, 6, 187 } % Color maricón

%-------------------------------------------------
%     MÁRGENES
%-------------------------------------------------

\usepackage{geometry}
\geometry{
    top=3cm,
    bottom=3cm,
    left=3cm,
    right=3cm,
    headheight=14pt,
	footskip=1.4cm,
	headsep=10pt,
}

\usepackage{avant} % Esto es una fuente para encabezados

%\usepackage{mathptmx} % Usar simbolitos matemáticos chulos

\usepackage{microtype} % Para fuentes de maricones

\usepackage[utf8]{inputenc} % Pa los acentos

\usepackage[T1]{fontenc}

%-------------------------------------------------
% Bibliografía e índice
%-------------------------------------------------

\usepackage{makeidx} % Pa hacer un índice
\makeindex

\usepackage{titletoc}   % Para manipular la tabla de contenidos

\contentsmargin{0cm}    % Para eliminar el margen por defecto

\usepackage{titlesec} % Pa cambiar los titulos skere

\titleformat
{\chapter} % command
[display] % shape
{\centering\bfseries\Huge\normalfont} % format
{\color{turquoise}  {\normalsize\MakeUppercase{Capítulo} \thechapter }} % label
{-0.5cm} % sep
{
    \color{turquoise}
    \rule{\textwidth}{3pt}
    \vspace{1ex}
    \centering
    \setcounter{ex}{0}
    \setcounter{dummy}{0}
} % before-code
[
\vspace{-0.5cm}%
\rule{\textwidth}{3pt}
] % after-code


\titleformat{\part}
[display]
{\centering\bfseries\Huge\normalfont}
{\color{turquoise} {\normalsize \MakeUppercase{Asignatura}}}
{0pt}
{\color{turquoise}
\vspace{-0.6cm}
\rule{\textwidth}{3pt}
\vspace{1ex}
\setcounter{chapter}{0}
\setcounter{section}{0}
\setcounter{dummy}{0}
\centering
}


\titleformat{\section}
{\normalfont\Large\bfseries}{\color{turquoise}\thesection\ - }{0.5em}{}

\usepackage{fancyhdr}   % Necesario para el encabezado y el pie de página

\pagestyle{fancy}   %Para modificar los encabezados
\fancyhf{}          %Para eliminar los encabezados y pies de página por defecto.
\fancyhead[LE,RO]{\sffamily\normalsize\thepage}
\fancyfoot[C]{Ampliación de Probabilidad}
%HACER

\usepackage{amsmath,amsfonts,amssymb,amsthm,cancel} % PARA LAS MATES

%   LINEA 199, HACER CAPULLADAS

\newtheoremstyle{turquoisebox}
{0pt} %Espacio encima
{0pt} %Espacio abajo
{\normalfont} % Fuente del cuerpo
{} % Cantidad de identado
{\small\ssfamily\color{turquoise}} % Fuente en la que pone "TEOREMA"
{:} % Puntuación tras el teorema
{0.25em} %Espacio tras el teorema
{\thmname{#1}\thmnumber{#2}} %No sé si esto funciona


\newcounter{dummy}
\newcounter{ex}
\newtheorem{teoremote}[dummy]{\color{turquoise}Teorema}
\newtheorem{propositiont}{\color{turquoise}Proposición}[section]
\newtheorem{lemmat}{\color{turquoise}Lema}[section]
\newtheorem{definitionT}{\color{turquoise}Definición}[section]
\newtheorem{exerciseT}[ex]{Ejercicio}
\newtheorem{examplote}[ex]{\color{turquoise}Ejemplo}
\newtheorem{methodT}[dummy]{\color{turquoise}Método}


\RequirePackage[framemethod=default]{mdframed} % Required for creating the theorem, definition, exercise and corollary boxes

%Caja de teoremas

\newmdenv[skipabove=7pt,
skipbelow=7pt,
backgroundcolor=black!5,
linecolor=turquoise,
innerleftmargin=5pt,
innerrightmargin=5pt,
innertopmargin=5pt,
leftmargin=0cm,
rightmargin=0cm,
linewidth=3pt,
innerbottommargin=5pt]{tBox}

\newmdenv[skipabove=7pt,
skipbelow=7pt,
backgroundcolor=black!5,
linecolor=turquoise,
innerleftmargin=5pt,
innerrightmargin=5pt,
innertopmargin=5pt,
leftmargin=0cm,
rightmargin=0cm,
linewidth=1pt,
innerbottommargin=5pt]{pBox}

\newmdenv[skipabove=7pt,
skipbelow=7pt,
backgroundcolor=violet!7,
linecolor=turquoise,
innerleftmargin=5pt,
innerrightmargin=5pt,
innertopmargin=5pt,
leftmargin=0cm,
rightmargin=0cm,
rightline=false,
topline=false,
bottomline=false,
linewidth=4pt,
innerbottommargin=5pt]{mBox}

\newmdenv[skipabove=7pt,
skipbelow=7pt,
rightline=false,
leftline=true,
topline=false,
bottomline=false,
linecolor=turquoise,
innerleftmargin=5pt,
innerrightmargin=5pt,
innertopmargin=0pt,
leftmargin=0cm,
rightmargin=0cm,
linewidth=4pt,
innerbottommargin=0pt]{dBox}

\newmdenv[skipabove=7pt,
skipbelow=7pt,
rightline=false,
leftline=true,
topline=false,
bottomline=false,
backgroundcolor=black!3,
linecolor=turquoise!50,
innerleftmargin=5pt,
innerrightmargin=5pt,
innertopmargin=0pt,
innerbottommargin=5pt,
leftmargin=0cm,
rightmargin=0cm,
linewidth=4pt]{eBox}

\newmdenv[skipabove=7pt,
skipbelow=7pt,
leftline=true,
topline=false,
rightline=false,
bottomline=false,
backgroundcolor=cyan!5,
linecolor=turquoise,
innerleftmargin=5pt,
innerrightmargin=5pt,
innertopmargin=0pt,
innerbottommargin=5pt,
leftmargin=0cm,
rightmargin=0cm,
linewidth=4pt]{exBox}

\newenvironment{theorem}{\begin{tBox}\begin{teoremote}}{\end{teoremote}\end{tBox}}
\newenvironment{proposition}{\begin{pBox}\begin{propositiont}}{\end{propositiont}\end{pBox}}
\newenvironment{lemma}{\begin{pBox}\begin{lemmat}}{\end{lemmat}\end{pBox}}
\newenvironment{method}{\begin{mBox}\begin{methodT}}{\end{methodT}\end{mBox}}
\newenvironment{definition}{\begin{dBox}\begin{definitionT}}{\end{definitionT}\end{dBox}}
\newenvironment{exercise}{\begin{eBox}\begin{exerciseT}}{\hfill{\color{black}}\end{exerciseT}\end{eBox}}
\newenvironment{example}{\begin{exBox}\begin{examplote}}{\end{examplote}\end{exBox}}
\newenvironment{demonstration}{\begin{flushright}
      \color{turquoise} \textbf{Demostración}
\end{flushright}
}{\begin{flushright}
  $\square$
\end{flushright}}

\usepackage{geometry}
\geometry{
    top=3cm,
    bottom=3cm,
    left=3cm,
    right=3cm,
    headheight=14pt, 
    footskip=1.4cm,
    headsep=10pt,
}
\usepackage{graphicx}
\title{Ejercicios de Teoría de la Probabilidad}
\author{Paco Mora Caselles}
\date{\today}


\begin{document}

\maketitle



% \setcounter{chapter}{2}
\chapter{Hoja 3}

\setcounter{ex}{2}

\begin{exercise}
    $$ F(x) = \frac{1}{24}(5xI_{[0,1)}(x)+(5x+3)I_{[1,2)}(x)+(5x+6)I_{[2,3)}(x)+24I_{[3,+\infty)}(x)) $$

    Sea $ D = \{1,2,3\} $, los puntos de la recta con probabilidad distinta de 0, y $ P(\{1\})  =P(\{2\}) = P(\{3\}) = \dfrac{3}{24} $ (recordemos que $ P(\{1\})  = F(1)-F(1^{-})$).

    Usando el procedimiento visto en la descomposición de Lebesgue: $ P(D) = \dfrac{9}{24}  \implies \alpha = \dfrac{9}{24} \implies F(x) = \alpha F_{d}(x) + (1-\alpha )F_{c}(x)$

    Además, tenemos que $ P_{d}(B) = \dfrac{1}{\alpha}P(B \cap D) $ entonces:

    $$ 
    F_{d}(x) = 
    \left\{
    \begin{array}{ll}
        \dfrac{24}{9} \cdot 0 = 0& x \in (- \infty ,0)\\
        \dfrac{24}{9} \cdot 0 = 0& x \in [0,1)\\
        \dfrac{24}{9} \cdot \dfrac{3}{24} = \dfrac{1}{3} & x \in [1,2)\\
        \dfrac{24}{9} \cdot \dfrac{6}{24} = \dfrac{2}{3} & x \in [2,3)\\
        \dfrac{24}{9} \cdot \dfrac{9}{24} = 1 & x \in [3,+\infty)\\

    \end{array}
    \right.
    $$

    Pasando a la parte continua, $ P_{c}(B) = \dfrac{1}{1- \alpha}P(B \cap D ^{c}) $ y $ F_{c}(x)=P_{c}((-\infty,x]) $\\$= \dfrac{1}{1-\alpha}P((-\infty,x]\cap D ^{c}) $

    $$ 
    F_{c}(x) = 
    \left\{
    \begin{array}{ll}
        \dfrac{24}{15} \cdot 0 & x \in (-\infty,0)\\
        \dfrac{24}{15} \cdot \dfrac{5x}{24} & x \in [0,1)\\
        \dfrac{24}{15} \cdot \left(\dfrac{5x+3}{24}-\underbrace{\dfrac{3}{24}}_{P(\{1\})}\right) & x \in [1,2)\\
        \dfrac{24}{15} \cdot \left(\dfrac{5x+6}{24}-\dfrac{6}{24}\right) & x \in [2,3)\\
        \dfrac{24}{15} \cdot \left(\dfrac{24}{24}\dfrac{9}{24}\right) & x \in [3,+\infty)
    \end{array}
    \right.
    $$


\end{exercise}

\chapter{Hoja 4}

\begin{exercise}
    $$ f_{X} (x) = 2(1-x) I_{(0,1)}(x) $$
    \textbf{a)} $ Y = aX-b $ con $ a \ne 0 $

    La función usada es la $ g(x) = ax-b $, esta función es continua, biyectiva (al ser monótona). Será creciente o decreciente dependiendo del valor de $ a $. Si $ h(y) = g ^{-1}(x) = \dfrac{y+b}{a} $, recordemos que:
    $$ f_{Y}(y) = f_{X}(h(y)) |h'(y)|\ si\ y \in g((0,1))\hspace{5mm}f_{Y}(y) = 0\ resto $$

    Calculamos $ h'(y) = \dfrac{1}{a} $ y $ g((0,1)) $:
    $$ g((0,1)) = \left\{
    \begin{array}{ll}
        (-b,a-b) & a>0\\
        (a-b,-b) & a<0
    \end{array}
    \right. $$
    
    Con lo que:
    $$ a > 0\hspace{5mm}f_{Y}(y) = 2\left(1-\dfrac{y+b}{a}\right)\dfrac{1}{a} = 2 \dfrac{a-y-b}{a^2}I_{(-b,a-b)} $$
    $$ a < 0\hspace{5mm}f_{Y}(y) = 2\left(1-\dfrac{y+b}{a}\right)-\dfrac{1}{a} = 2 \dfrac{y+b-a}{a^2}I_{(a-b,-b)} $$
    \textbf{b)} $ Z = 3X^2-X $

    Usaremos la función $ g(x) = 3x^2-x $, esta función no es biyectiva, tendremos que usar dos intervalos $ E_1,E_2 $ para hacer el cambio de variable.


    En primer lugar, vemos que el mínimo de la parábola está en $ x = \dfrac{1}{6} $, con lo que tenemos los conjuntos $ E_1 = \left(0,\dfrac{1}{6}\right),\ E_2 = \left(\dfrac{1}{6},1\right) $, tenemos que:
    $$ E_1 \to \left(-\dfrac{1}{12},0\right) = F_1 \hspace{5mm}E_2 = \left(\dfrac{1}{6},1\right) \to \left(-\dfrac{1}{12},2\right) = F_2 $$

    Para cada intervalo, definimos $ g_i $:
    $$ g_1 = g|_{(0,\frac{1}{6})}:\left(0,\dfrac{1}{6}\right) \to \left(-\dfrac{1}{12},0\right) \hspace{1cm} g_2 = g|_{(\frac{1}{6},1)}:\left(\dfrac{1}{6},1\right) \to \left(-\dfrac{1}{12},2\right)$$

    Entonces tenemos que:
    $$ f_{Z}(z) = \sum\limits_{r}^{} f_{X}(h_{r}(z)) |h'_{r}(z)|  $$
    
    Siendo $ h_{r}(z) $ la inversa de $ g_{r}(z) $, las calculamos:
    $$ z = 3x^2-x \iff 3x^2-x-z = 0 \iff x = \dfrac{1 \pm \sqrt{1+12z}}{6} = \left\{
    \begin{array}{lr}
        \dfrac{1+\sqrt{1+12z}}{6} = h_2(z) & (creciente)\\
        \dfrac{1-\sqrt{1+12z}}{6} = h_1(z) & (decreciente)
    \end{array}
    \right. $$

    $$ h_1'(z) = -\dfrac{12}{26 \sqrt{1+12z}} = -\dfrac{1}{\sqrt{1+12z}} $$
    $$ h_2'(z) = \dfrac{1}{\sqrt{1+12z}} $$

    Entonces tenemos que:
    $$z \in \left(-\dfrac{1}{12},0\right) \implies f_{Z}(z) = 2\left(1-\dfrac{1-\sqrt{1+12z}}{6}\right) \dfrac{1}{\sqrt{1+12z}} + 2\left(1-\dfrac{1+\sqrt{1+12z}}{6}\right) \dfrac{1}{\sqrt{1+12z}} =  $$
    $$ = 2 \dfrac{1}{\sqrt{1+12z}} \left( 2-\dfrac{2}{6} \right) = \dfrac{2 \cdot 10}{6\sqrt{1+12z}} = \dfrac{10}{3 \sqrt{1+12z}} $$

    $$ z \in (0,2) \implies f_{Z}(z) = 2\left(1-\dfrac{1+\sqrt{1+12z}}{6}\right) \dfrac{1}{\sqrt{1+12z}} $$


\end{exercise}

\begin{exercise}
    $$ f(x,y) = \dfrac{2}{(2-x-y)^3}I_{E}(x,y)\hspace{5mm} \text{con $ E $ el cuadrilátero de vértices (0,0),(1,0),($ \dfrac{2}{3},\dfrac{2}{3} $),(0,1)} $$

    Y los cambios de variable:

    $$ \left\{
    \begin{array}{l}
        U = \dfrac{X}{2-X-Y}\\
        V = \dfrac{Y}{2-X-Y}
    \end{array}
    \right. $$

    $$ \left\{
    \begin{array}{l}
        u(x,y) = \dfrac{x}{2-x-y}\\
        v(x,y) = \dfrac{y}{2-x-y}
    \end{array}
    \right. $$

    Esta transformación es biyectiva.
    
    Como comentario, recordar que los cambios de variable de la forma:

    $$ u = \dfrac{ax+by+c}{dx+ey+f} \hspace{5mm} v = \dfrac{a'x+b'y+c'}{dx+ey+f} $$

    Además de ser biyectivos transforman rectas en rectas.

    Entonces tenemos que:
    $$ f_{U,V} (u,v) = f(x(u,v),y(u,v)) \left| \dfrac{\partial (x,y)}{\partial(u,v)} \right|$$

    Vemos en qué se transforman los vértices del cuadrilátero con estas transformaciones:
    \begin{enumerate}
        \item $ (0,0) \to (0,0) $
        \item $ (1,0) \to (1,0) $
        \item $ ( \dfrac{2}{3},\dfrac{2}{3}) \to (1,1)$
        \item $ (0,1) \to (0,1) $
    \end{enumerate}

    Calculamos ahora las inversas:

    $$ u(2-x-y) = x \iff -2u+ux+uy+x = 0 \iff (1+u)x+uy-2u = 0 $$
    $$ v(2-x-y) = v \iff -2v+vx+vy+y = 0 \iff (1+v)y+ux-2v = 0 $$

    Espectacular sistema de ecuaciones lineales del que sacamos que $ x = \dfrac{2u}{1+u+v}\ y = \dfrac{2v}{1+u+v} $

    Ahora,
    $$ \dfrac{\partial x}{\partial u} = \dfrac{2(1+u+v)-2u}{(1+u+v)^2} = \dfrac{2(1+v)}{(1+u+v)^2} $$
    $$ \dfrac{\partial x}{\partial v} = -\dfrac{2u}{(1+u+v)^2} $$
    $$ \dfrac{\partial y}{\partial u} = -\dfrac{2v}{(1+u+v)^2} $$
    $$ \dfrac{\partial y}{\partial v} = \dfrac{2(1+u)}{(1+u+v)^2} $$

    El Jacobiano entonces es $ \dfrac{4}{(1+u+v)^3} $, con lo que la función de densidad $ f_{(U,V)} $ es:
    $$ f_{(U,V)}(u,v) = \dfrac{2}{\left(2-\dfrac{2u}{1+u+v}\right)^3} \dfrac{4}{\left( 1+u+v \right)^3} = 1 \hspace{5mm} Si\ (u,v)\in (0,1)\times(0,1) $$
    $$ f_{(U,V)}(u,v) = 0\ Si (u,v) \not \in (0,1) \times (0,1) $$

    

\end{exercise}


\end{document}