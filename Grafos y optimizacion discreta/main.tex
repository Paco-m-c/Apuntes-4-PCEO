\documentclass[openany]{book}
\usepackage[utf8]{inputenc}
\usepackage{verbatim}
\usepackage[hypertexnames=false]{hyperref}
\usepackage{amstext} 
\usepackage{array}   
\newcolumntype{C}{>{$}c<{$}} 


%%%%%%%%%%%%%%%%%%%%%%%

%%%%%%%%%%%%%%%%%%%%%%%
% HOLA PACO
% ESTE ES EL ARCHIVO DE LAS DEFINICIONES ESTRUCTURALES
% VERSION 1.1 NOMÁS
%
% AUTOR ORIGINAL:
% EDUARDO (CHITO) BELMONTE GUILLAMÓN
%
% ESTE ARCHIVO ES COMUNISTA, PUEDES COMPARTIRLO SI QUIERES
%%%%%%%%%%%%%%%%%%%%%%%

%----------------------------------
%     PAQUETICOS QUE SE USAN
%----------------------------------

%--------------------------
%    PARA USAR INKSCAPE
%---------------------------
\usepackage{import}
\usepackage{hyperref}
\usepackage{xifthen}
\usepackage{pdfpages}
\usepackage{transparent}

\newcommand{\incfig}[1]{%
    \def\svgwidth{\columnwidth}
    \import{./figures/}{#1.pdf_tex}
}

\newcommand{\custincfig}[2]{%
    \def\svgwidth{#1}
    \import{./figures/}{#2.pdf_tex}
}
\newcommand{\textnexttofig}[3]{
  \begin{minipage}[l]{0.45\textwidth}
    \custincfig{#1}{#2}
  \end{minipage}
  \begin{minipage}[l]{0.45\textwidth}
    #3
  \end{minipage}
}

%%%%%%%%% FIN DEL INKSCAPE

\usepackage{parskip} % Pa parrafos wapos
\setlength{\parindent}{0.5cm} % Pa la sangría
\usepackage{graphicx} % Pa meter las imágenes
\graphicspath{{Images/}} % La ruta a las imágenes

\usepackage{tikz} % Pa dibujar cosichuelas guapas

\usepackage[spanish]{babel} % PA QUE ESTÉ EN ESPAÑOL NOMÁS

\usepackage{enumitem} % Para personalizar las LISTAS YEAH

\setlist{nolistsep} % Pa que las listas estén junticas

\usepackage{booktabs} % Esta sirve para hacer tablas fancy con multicolumns y tal pero no tengo ni puta idea de usarla

\usepackage{xcolor} % PA DEFINIR LOS COLORINES
\definecolor{turquoise}{RGB}{21,103,112} % Es un turquesica así formal
\definecolor{violet}{RGB}{ 110, 6, 187 } % Color maricón

%-------------------------------------------------
%     MÁRGENES
%-------------------------------------------------

\usepackage{geometry}
\geometry{
    top=3cm,
    bottom=3cm,
    left=3cm,
    right=3cm,
    headheight=14pt,
	footskip=1.4cm,
	headsep=10pt,
}

\usepackage{avant} % Esto es una fuente para encabezados

%\usepackage{mathptmx} % Usar simbolitos matemáticos chulos

\usepackage{microtype} % Para fuentes de maricones

\usepackage[utf8]{inputenc} % Pa los acentos

\usepackage[T1]{fontenc}

%-------------------------------------------------
% Bibliografía e índice
%-------------------------------------------------

\usepackage{makeidx} % Pa hacer un índice
\makeindex

\usepackage{titletoc}   % Para manipular la tabla de contenidos

\contentsmargin{0cm}    % Para eliminar el margen por defecto

\usepackage{titlesec} % Pa cambiar los titulos skere

\titleformat
{\chapter} % command
[display] % shape
{\centering\bfseries\Huge\normalfont} % format
{\color{turquoise}  {\normalsize\MakeUppercase{Capítulo} \thechapter }} % label
{-0.5cm} % sep
{
    \color{turquoise}
    \rule{\textwidth}{3pt}
    \vspace{1ex}
    \centering
    \setcounter{ex}{0}
    \setcounter{dummy}{0}
} % before-code
[
\vspace{-0.5cm}%
\rule{\textwidth}{3pt}
] % after-code


\titleformat{\part}
[display]
{\centering\bfseries\Huge\normalfont}
{\color{turquoise} {\normalsize \MakeUppercase{Asignatura}}}
{0pt}
{\color{turquoise}
\vspace{-0.6cm}
\rule{\textwidth}{3pt}
\vspace{1ex}
\setcounter{chapter}{0}
\setcounter{section}{0}
\setcounter{dummy}{0}
\centering
}


\titleformat{\section}
{\normalfont\Large\bfseries}{\color{turquoise}\thesection\ - }{0.5em}{}

\usepackage{fancyhdr}   % Necesario para el encabezado y el pie de página

\pagestyle{fancy}   %Para modificar los encabezados
\fancyhf{}          %Para eliminar los encabezados y pies de página por defecto.
\fancyhead[LE,RO]{\sffamily\normalsize\thepage}
\fancyfoot[C]{Ampliación de Probabilidad}
%HACER

\usepackage{amsmath,amsfonts,amssymb,amsthm,cancel} % PARA LAS MATES

%   LINEA 199, HACER CAPULLADAS

\newtheoremstyle{turquoisebox}
{0pt} %Espacio encima
{0pt} %Espacio abajo
{\normalfont} % Fuente del cuerpo
{} % Cantidad de identado
{\small\ssfamily\color{turquoise}} % Fuente en la que pone "TEOREMA"
{:} % Puntuación tras el teorema
{0.25em} %Espacio tras el teorema
{\thmname{#1}\thmnumber{#2}} %No sé si esto funciona


\newcounter{dummy}
\newcounter{ex}
\newtheorem{teoremote}[dummy]{\color{turquoise}Teorema}
\newtheorem{propositiont}{\color{turquoise}Proposición}[section]
\newtheorem{lemmat}{\color{turquoise}Lema}[section]
\newtheorem{definitionT}{\color{turquoise}Definición}[section]
\newtheorem{exerciseT}[ex]{Ejercicio}
\newtheorem{examplote}[ex]{\color{turquoise}Ejemplo}
\newtheorem{methodT}[dummy]{\color{turquoise}Método}


\RequirePackage[framemethod=default]{mdframed} % Required for creating the theorem, definition, exercise and corollary boxes

%Caja de teoremas

\newmdenv[skipabove=7pt,
skipbelow=7pt,
backgroundcolor=black!5,
linecolor=turquoise,
innerleftmargin=5pt,
innerrightmargin=5pt,
innertopmargin=5pt,
leftmargin=0cm,
rightmargin=0cm,
linewidth=3pt,
innerbottommargin=5pt]{tBox}

\newmdenv[skipabove=7pt,
skipbelow=7pt,
backgroundcolor=black!5,
linecolor=turquoise,
innerleftmargin=5pt,
innerrightmargin=5pt,
innertopmargin=5pt,
leftmargin=0cm,
rightmargin=0cm,
linewidth=1pt,
innerbottommargin=5pt]{pBox}

\newmdenv[skipabove=7pt,
skipbelow=7pt,
backgroundcolor=violet!7,
linecolor=turquoise,
innerleftmargin=5pt,
innerrightmargin=5pt,
innertopmargin=5pt,
leftmargin=0cm,
rightmargin=0cm,
rightline=false,
topline=false,
bottomline=false,
linewidth=4pt,
innerbottommargin=5pt]{mBox}

\newmdenv[skipabove=7pt,
skipbelow=7pt,
rightline=false,
leftline=true,
topline=false,
bottomline=false,
linecolor=turquoise,
innerleftmargin=5pt,
innerrightmargin=5pt,
innertopmargin=0pt,
leftmargin=0cm,
rightmargin=0cm,
linewidth=4pt,
innerbottommargin=0pt]{dBox}

\newmdenv[skipabove=7pt,
skipbelow=7pt,
rightline=false,
leftline=true,
topline=false,
bottomline=false,
backgroundcolor=black!3,
linecolor=turquoise!50,
innerleftmargin=5pt,
innerrightmargin=5pt,
innertopmargin=0pt,
innerbottommargin=5pt,
leftmargin=0cm,
rightmargin=0cm,
linewidth=4pt]{eBox}

\newmdenv[skipabove=7pt,
skipbelow=7pt,
leftline=true,
topline=false,
rightline=false,
bottomline=false,
backgroundcolor=cyan!5,
linecolor=turquoise,
innerleftmargin=5pt,
innerrightmargin=5pt,
innertopmargin=0pt,
innerbottommargin=5pt,
leftmargin=0cm,
rightmargin=0cm,
linewidth=4pt]{exBox}

\newenvironment{theorem}{\begin{tBox}\begin{teoremote}}{\end{teoremote}\end{tBox}}
\newenvironment{proposition}{\begin{pBox}\begin{propositiont}}{\end{propositiont}\end{pBox}}
\newenvironment{lemma}{\begin{pBox}\begin{lemmat}}{\end{lemmat}\end{pBox}}
\newenvironment{method}{\begin{mBox}\begin{methodT}}{\end{methodT}\end{mBox}}
\newenvironment{definition}{\begin{dBox}\begin{definitionT}}{\end{definitionT}\end{dBox}}
\newenvironment{exercise}{\begin{eBox}\begin{exerciseT}}{\hfill{\color{black}}\end{exerciseT}\end{eBox}}
\newenvironment{example}{\begin{exBox}\begin{examplote}}{\end{examplote}\end{exBox}}
\newenvironment{demonstration}{\begin{flushright}
      \color{turquoise} \textbf{Demostración}
\end{flushright}
}{\begin{flushright}
  $\square$
\end{flushright}}

\usepackage{geometry}
\geometry{
    top=3cm,
    bottom=3cm,
    left=3cm,
    right=3cm,
    headheight=14pt, 
    footskip=1.4cm,
    headsep=10pt,
}
\usepackage{graphicx}
\title{Apuntes de Grafos}
\author{Paco Mora\\Manuel Franco}
\date{\today}

\begin{document}

\maketitle
\chapter{Tema 3. Árboles}
\begin{definition}
    Diremos que un grafo $ G = (V,E)$ es un árbol si es conexo y no tiene ciclos.\\
    Un árbol generador de un grafo $ G= (V,E) $ es un subgrafo parcial conexo y sin ciclos.\\
    Un bosque es un grafo $ G = (V,E) $ sin ciclos. 
\end{definition}


\begin{definition}
    En un árbol, los nodos con grado de incidencia 1 se denominan hojas.
\end{definition}

\begin{theorem}
    Teorema de caracterización de árboles
    Sea $ G = (V,E) $. Son equivalentes:
    \begin{itemize}
        \item $ G $ es conexo y sin ciclos
        \item Entre cada par de vértices distintos de $ V $, existe una única cadena.
        \item $ G $ es conexo y $ m = n-1 $
        \item $ G $ no contiene ciclos y $ m = n-1 $
        \item $ G $ está minimalmente conectado
        \item $ G $ no contiene ciclos y su añadimos una arista entre dos vértices no adyacentes cualesquiera de $ V $, el grafo que se obtiene contiene un único ciclo. 
    \end{itemize}
\end{theorem}

\begin{demonstration}
    $ 1 \implies 2 $\\

    $ G $ es conexo sin nodos $ \implies  \forall u \ne v \exists! $ cadena $ u ~ v $. Existe una cadena por ser conexo, la yuxtaposición de dos cadenas diferentes $ u ~ v $, $  G $contendría al menos un ciclo.\\
    $ 2 \implies 3 $\\
    
    Suponemos que existe una única cadena entre cada par de vértices $ u,v $. Como existe una cadena entre cada par de vértices, $ G $ es conexo. Veamos que $ m = n-1 $. Recordemos una proposición que decía:

    ''Si $ G $ es conexo $ m \geq n-1 $''

    Veamos la igualdad ahora por inducción sobre el número de nodos, el caso $ n = 1,2 $ es directo.\\
    Si $ n>2 $, eliminamos una arista cualquiera del grafo: $ e =(u,v)$. Dado que esa cadena ($ u,(u,v),v $) era la única que conectaba $ u,v $, ahora estos vértices están en componentes conexas distintas, con $ n_1,n_2 $ nodos y $ m_1,m_2 $ aristas respectivamente, que siguen cumpliendo la hipótesis de inducción, luego $ m_1 = n_1-1 $ y $ m_2 =n_2-1 $. En $ G $, $ n = n_1+n_2=m_1+1+m_2+1 = (m_1+m_2+1) +1 = m+1 $\\
    $ 3 \implies 4 $\\

    $G$ conexo y $m = n-1$ $\implies$ $G$ no contiene ciclos y $ m = n-1 $

    Supongamos que $G$ contiene un ciclo y retiráramos una arista cualquiera $e$ no desconectaría el grafo y tendría un grafo conexo con $n$ nodos y $(n-1) - 1$ aristas, por la proposición que hemos recordado antes, $ G $ no sería conexo, lo que contradice (3)\\
    $ 4 \implies 5 $\\

    $G$ no tiene ciclos y $m=n-1$ $\implies$ $G$ está minimalmente conectado. Por la proposición que hemos recordado antes, basta demostrar que $G$ es conexo.

    Supongamos que $G$ contiene $s$ componentes conexas : $ (V_1,E_1),...,(V_s,E_s) $ con $ n_i $ nodos y $ m_i $ aristas, tengo ahora que $G$ es acíclico, por lo que cada cc es acíclica y conexa por lo que cumple $1$, y por tanto $3$, y por tanto cada $m_{i}= n_i-1$

    $$ (3) \implies m_i = n{i-1} \forall i\ n = \sum\limits_{i=1}^{s}n_i = \sum\limits_{i=1}^{s}(m_i+1) = \sum\limits_{i=1}^{s}m_i+s = m+s $$
    Como partiamos de que $n = m + 1$ y tenemos $n = m + s$, entonces $s=1$ y hay solo una $c^3$.
    
    $ 5 \implies 6 $\\ 


\end{demonstration}



% A partir de aqui son cosas mienras el ordenador de chito enciende


\begin{theorem}
    {\textbf{\color{turquoise}Algoritmo de Kruskal}}


    \textbf{Paso 1}

    Ordenar las aristas de $ E $ en orden ascendente de su peso:
    $$ V = \{v_1,...,v_n\},\ T^* = (V, \emptyset) $$
    $$ E := \{e_1,...,e_m\}:\ \mathcal{l} \leq \mathcal{l}(e_i+1) \forall i < m $$

    \textbf{Paso 2}

    Añadir $ n-1 $ aristas a $ T^* $ sucesivamente (en el orden de sus pesos) sin que se formen ciclos.
\end{theorem}


\begin{theorem}
    {\textbf{\color{turquoise}Algoritmo de Prim}}

    \textbf{Paso 1}

    Elegir un vértice $ r \in V $ y hacer $ V_1 = \{r\},\ V_2 = V \setminus \{r\} $.

    \textbf{Paso 2}

    Añadir al árbol la arista de menor peso de $ w(V_1) $, digamos $ (v_1,v_2) $ con $ v_1 \in V_1 $ y $ v_2 \in V_2 $. Añadir $ v_2  $ a $ V_1 $ y borrar $ v_2  $ de $ V_2 $.

    \textbf{Paso 3}

    Si $ |V_1| = n $ parar. Si no, volver al Paso 2.



\end{theorem}


\end{document}