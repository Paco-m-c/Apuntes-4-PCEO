\documentclass[openany]{book}
\usepackage[utf8]{inputenc}
\usepackage{verbatim}
\usepackage[hypertexnames=false]{hyperref}
\usepackage{amstext} 
\usepackage{array}   
\newcolumntype{C}{>{$}c<{$}} 


%%%%%%%%%%%%%%%%%%%%%%%

%%%%%%%%%%%%%%%%%%%%%%%
% HOLA PACO
% ESTE ES EL ARCHIVO DE LAS DEFINICIONES ESTRUCTURALES
% VERSION 1.1 NOMÁS
%
% AUTOR ORIGINAL:
% EDUARDO (CHITO) BELMONTE GUILLAMÓN
%
% ESTE ARCHIVO ES COMUNISTA, PUEDES COMPARTIRLO SI QUIERES
%%%%%%%%%%%%%%%%%%%%%%%

%----------------------------------
%     PAQUETICOS QUE SE USAN
%----------------------------------

%--------------------------
%    PARA USAR INKSCAPE
%---------------------------
\usepackage{import}
\usepackage{hyperref}
\usepackage{xifthen}
\usepackage{pdfpages}
\usepackage{transparent}

\newcommand{\incfig}[1]{%
    \def\svgwidth{\columnwidth}
    \import{./figures/}{#1.pdf_tex}
}

\newcommand{\custincfig}[2]{%
    \def\svgwidth{#1}
    \import{./figures/}{#2.pdf_tex}
}
\newcommand{\textnexttofig}[3]{
  \begin{minipage}[l]{0.45\textwidth}
    \custincfig{#1}{#2}
  \end{minipage}
  \begin{minipage}[l]{0.45\textwidth}
    #3
  \end{minipage}
}

%%%%%%%%% FIN DEL INKSCAPE

\usepackage{parskip} % Pa parrafos wapos
\setlength{\parindent}{0.5cm} % Pa la sangría
\usepackage{graphicx} % Pa meter las imágenes
\graphicspath{{Images/}} % La ruta a las imágenes

\usepackage{tikz} % Pa dibujar cosichuelas guapas

\usepackage[spanish]{babel} % PA QUE ESTÉ EN ESPAÑOL NOMÁS

\usepackage{enumitem} % Para personalizar las LISTAS YEAH

\setlist{nolistsep} % Pa que las listas estén junticas

\usepackage{booktabs} % Esta sirve para hacer tablas fancy con multicolumns y tal pero no tengo ni puta idea de usarla

\usepackage{xcolor} % PA DEFINIR LOS COLORINES
\definecolor{turquoise}{RGB}{21,103,112} % Es un turquesica así formal
\definecolor{violet}{RGB}{ 110, 6, 187 } % Color maricón

%-------------------------------------------------
%     MÁRGENES
%-------------------------------------------------

\usepackage{geometry}
\geometry{
    top=3cm,
    bottom=3cm,
    left=3cm,
    right=3cm,
    headheight=14pt,
	footskip=1.4cm,
	headsep=10pt,
}

\usepackage{avant} % Esto es una fuente para encabezados

%\usepackage{mathptmx} % Usar simbolitos matemáticos chulos

\usepackage{microtype} % Para fuentes de maricones

\usepackage[utf8]{inputenc} % Pa los acentos

\usepackage[T1]{fontenc}

%-------------------------------------------------
% Bibliografía e índice
%-------------------------------------------------

\usepackage{makeidx} % Pa hacer un índice
\makeindex

\usepackage{titletoc}   % Para manipular la tabla de contenidos

\contentsmargin{0cm}    % Para eliminar el margen por defecto

\usepackage{titlesec} % Pa cambiar los titulos skere

\titleformat
{\chapter} % command
[display] % shape
{\centering\bfseries\Huge\normalfont} % format
{\color{turquoise}  {\normalsize\MakeUppercase{Capítulo} \thechapter }} % label
{-0.5cm} % sep
{
    \color{turquoise}
    \rule{\textwidth}{3pt}
    \vspace{1ex}
    \centering
    \setcounter{ex}{0}
    \setcounter{dummy}{0}
} % before-code
[
\vspace{-0.5cm}%
\rule{\textwidth}{3pt}
] % after-code


\titleformat{\part}
[display]
{\centering\bfseries\Huge\normalfont}
{\color{turquoise} {\normalsize \MakeUppercase{Asignatura}}}
{0pt}
{\color{turquoise}
\vspace{-0.6cm}
\rule{\textwidth}{3pt}
\vspace{1ex}
\setcounter{chapter}{0}
\setcounter{section}{0}
\setcounter{dummy}{0}
\centering
}


\titleformat{\section}
{\normalfont\Large\bfseries}{\color{turquoise}\thesection\ - }{0.5em}{}

\usepackage{fancyhdr}   % Necesario para el encabezado y el pie de página

\pagestyle{fancy}   %Para modificar los encabezados
\fancyhf{}          %Para eliminar los encabezados y pies de página por defecto.
\fancyhead[LE,RO]{\sffamily\normalsize\thepage}
\fancyfoot[C]{Ampliación de Probabilidad}
%HACER

\usepackage{amsmath,amsfonts,amssymb,amsthm,cancel} % PARA LAS MATES

%   LINEA 199, HACER CAPULLADAS

\newtheoremstyle{turquoisebox}
{0pt} %Espacio encima
{0pt} %Espacio abajo
{\normalfont} % Fuente del cuerpo
{} % Cantidad de identado
{\small\ssfamily\color{turquoise}} % Fuente en la que pone "TEOREMA"
{:} % Puntuación tras el teorema
{0.25em} %Espacio tras el teorema
{\thmname{#1}\thmnumber{#2}} %No sé si esto funciona


\newcounter{dummy}
\newcounter{ex}
\newtheorem{teoremote}[dummy]{\color{turquoise}Teorema}
\newtheorem{propositiont}{\color{turquoise}Proposición}[section]
\newtheorem{lemmat}{\color{turquoise}Lema}[section]
\newtheorem{definitionT}{\color{turquoise}Definición}[section]
\newtheorem{exerciseT}[ex]{Ejercicio}
\newtheorem{examplote}[ex]{\color{turquoise}Ejemplo}
\newtheorem{methodT}[dummy]{\color{turquoise}Método}


\RequirePackage[framemethod=default]{mdframed} % Required for creating the theorem, definition, exercise and corollary boxes

%Caja de teoremas

\newmdenv[skipabove=7pt,
skipbelow=7pt,
backgroundcolor=black!5,
linecolor=turquoise,
innerleftmargin=5pt,
innerrightmargin=5pt,
innertopmargin=5pt,
leftmargin=0cm,
rightmargin=0cm,
linewidth=3pt,
innerbottommargin=5pt]{tBox}

\newmdenv[skipabove=7pt,
skipbelow=7pt,
backgroundcolor=black!5,
linecolor=turquoise,
innerleftmargin=5pt,
innerrightmargin=5pt,
innertopmargin=5pt,
leftmargin=0cm,
rightmargin=0cm,
linewidth=1pt,
innerbottommargin=5pt]{pBox}

\newmdenv[skipabove=7pt,
skipbelow=7pt,
backgroundcolor=violet!7,
linecolor=turquoise,
innerleftmargin=5pt,
innerrightmargin=5pt,
innertopmargin=5pt,
leftmargin=0cm,
rightmargin=0cm,
rightline=false,
topline=false,
bottomline=false,
linewidth=4pt,
innerbottommargin=5pt]{mBox}

\newmdenv[skipabove=7pt,
skipbelow=7pt,
rightline=false,
leftline=true,
topline=false,
bottomline=false,
linecolor=turquoise,
innerleftmargin=5pt,
innerrightmargin=5pt,
innertopmargin=0pt,
leftmargin=0cm,
rightmargin=0cm,
linewidth=4pt,
innerbottommargin=0pt]{dBox}

\newmdenv[skipabove=7pt,
skipbelow=7pt,
rightline=false,
leftline=true,
topline=false,
bottomline=false,
backgroundcolor=black!3,
linecolor=turquoise!50,
innerleftmargin=5pt,
innerrightmargin=5pt,
innertopmargin=0pt,
innerbottommargin=5pt,
leftmargin=0cm,
rightmargin=0cm,
linewidth=4pt]{eBox}

\newmdenv[skipabove=7pt,
skipbelow=7pt,
leftline=true,
topline=false,
rightline=false,
bottomline=false,
backgroundcolor=cyan!5,
linecolor=turquoise,
innerleftmargin=5pt,
innerrightmargin=5pt,
innertopmargin=0pt,
innerbottommargin=5pt,
leftmargin=0cm,
rightmargin=0cm,
linewidth=4pt]{exBox}

\newenvironment{theorem}{\begin{tBox}\begin{teoremote}}{\end{teoremote}\end{tBox}}
\newenvironment{proposition}{\begin{pBox}\begin{propositiont}}{\end{propositiont}\end{pBox}}
\newenvironment{lemma}{\begin{pBox}\begin{lemmat}}{\end{lemmat}\end{pBox}}
\newenvironment{method}{\begin{mBox}\begin{methodT}}{\end{methodT}\end{mBox}}
\newenvironment{definition}{\begin{dBox}\begin{definitionT}}{\end{definitionT}\end{dBox}}
\newenvironment{exercise}{\begin{eBox}\begin{exerciseT}}{\hfill{\color{black}}\end{exerciseT}\end{eBox}}
\newenvironment{example}{\begin{exBox}\begin{examplote}}{\end{examplote}\end{exBox}}
\newenvironment{demonstration}{\begin{flushright}
      \color{turquoise} \textbf{Demostración}
\end{flushright}
}{\begin{flushright}
  $\square$
\end{flushright}}

\usepackage{geometry}
\geometry{
    top=3cm,
    bottom=3cm,
    left=3cm,
    right=3cm,
    headheight=14pt, 
    footskip=1.4cm,
    headsep=10pt,
}
\usepackage{graphicx}
\title{Apuntes de Grafos}
\author{Paco Mora\\Manuel Franco}
\date{\today}

\begin{document}

\maketitle
\chapter{Tema 3. Árboles}
\begin{definition}
    Diremos que un grafo $ G = (V,E)$ es un árbol si es conexo y no tiene ciclos.\\
    Un árbol generador de un grafo $ G= (V,E) $ es un subgrafo parcial conexo y sin ciclos.\\
    Un bosque es un grafo $ G = (V,E) $ sin ciclos. 
\end{definition}


\begin{definition}
    En un árbol, los nodos con grado de incidencia 1 se denominan hojas.
\end{definition}

\begin{theorem}
    Teorema de caracterización de árboles
    Sea $ G = (V,E) $. Son equivalentes:
    \begin{itemize}
        \item $ G $ es conexo y sin ciclos
        \item Entre cada par de vértices distintos de $ V $, existe una única cadena.
        \item $ G $ es conexo y $ m = n-1 $
        \item $ G $ no contiene ciclos y $ m = n-1 $
        \item $ G $ está minimalmente conectado
        \item $ G $ no contiene ciclos y su añadimos una arista entre dos vértices no adyacentes cualesquiera de $ V $, el grafo que se obtiene contiene un único ciclo. 
    \end{itemize}
\end{theorem}

\begin{demonstration}
    $ 1 \implies 2 $\\

    $ G $ es conexo sin nodos $ \implies  \forall u \ne v \exists! $ cadena $ u ~ v $. Existe una cadena por ser conexo, la yuxtaposición de dos cadenas diferentes $ u ~ v $, $  G $contendría al menos un ciclo.\\
    $ 2 \implies 3 $\\
    
    Suponemos que existe una única cadena entre cada par de vértices $ u,v $. Como existe una cadena entre cada par de vértices, $ G $ es conexo. Veamos que $ m = n-1 $. Recordemos una proposición que decía:

    ''Si $ G $ es conexo $ m \geq n-1 $''

    Veamos la igualdad ahora por inducción sobre el número de nodos, el caso $ n = 1,2 $ es directo.\\
    Si $ n>2 $, eliminamos una arista cualquiera del grafo: $ e =(u,v)$. Dado que esa cadena ($ u,(u,v),v $) era la única que conectaba $ u,v $, ahora estos vértices están en componentes conexas distintas, con $ n_1,n_2 $ nodos y $ m_1,m_2 $ aristas respectivamente, que siguen cumpliendo la hipótesis de inducción, luego $ m_1 = n_1-1 $ y $ m_2 =n_2-1 $. En $ G $, $ n = n_1+n_2=m_1+1+m_2+1 = (m_1+m_2+1) +1 = m+1 $\\
    $ 3 \implies 4 $\\

    $G$ conexo y $m = n-1$ $\implies$ $G$ no contiene ciclos y $ m = n-1 $

    Supongamos que $G$ contiene un ciclo y retiráramos una arista cualquiera $e$ no desconectaría el grafo y tendría un grafo conexo con $n$ nodos y $(n-1) - 1$ aristas, por la proposición que hemos recordado antes, $ G $ no sería conexo, lo que contradice (3)\\
    $ 4 \implies 5 $\\

    $G$ no tiene ciclos y $m=n-1$ $\implies$ $G$ está minimalmente conectado. Por la proposición que hemos recordado antes, basta demostrar que $G$ es conexo.

    Supongamos que $G$ contiene $s$ componentes conexas : $ (V_1,E_1),...,(V_s,E_s) $ con $ n_i $ nodos y $ m_i $ aristas, tengo ahora que $G$ es acíclico, por lo que cada cc es acíclica y conexa por lo que cumple $1$, y por tanto $3$, y por tanto cada $m_{i}= n_i-1$

    $$ (3) \implies m_i = n{i-1} \forall i\ n = \sum\limits_{i=1}^{s}n_i = \sum\limits_{i=1}^{s}(m_i+1) = \sum\limits_{i=1}^{s}m_i+s = m+s $$
    Como partiamos de que $n = m + 1$ y tenemos $n = m + s$, entonces $s=1$ y hay solo una $c^3$.
    
    $ 5 \implies 6 $\\ 


\end{demonstration}



% A partir de aqui son cosas mientras el ordenador de chito enciende


\begin{theorem}
    {\textbf{\color{turquoise}Algoritmo de Kruskal}}


    \textbf{Paso 1}

    Ordenar las aristas de $ E $ en orden ascendente de su peso:
    $$ V = \{v_1,...,v_n\},\ T^* = (V, \emptyset) $$
    $$ E := \{e_1,...,e_m\}:\ \mathcal{l} \leq \mathcal{l}(e_i+1) \forall i < m $$

    \textbf{Paso 2}

    Añadir $ n-1 $ aristas a $ T^* $ sucesivamente (en el orden de sus pesos) sin que se formen ciclos.
\end{theorem}


\begin{theorem}
    {\textbf{\color{turquoise}Algoritmo de Prim}}

    \textbf{Paso 1}

    Elegir un vértice $ r \in V $ y hacer $ V_1 = \{r\},\ V_2 = V \setminus \{r\} $.

    \textbf{Paso 2}

    Añadir al árbol la arista de menor peso de $ w(V_1) $, digamos $ (v_1,v_2) $ con $ v_1 \in V_1 $ y $ v_2 \in V_2 $. Añadir $ v_2  $ a $ V_1 $ y borrar $ v_2  $ de $ V_2 $.

    \textbf{Paso 3}

    Si $ |V_1| = n $ parar. Si no, volver al Paso 2.



\end{theorem}




%%%22/10 ESTO ES PARTE DEL TEMA 6

\section{Problemas de optimización sobre grafos}

\begin{exercise}
    {\color{turquoise} \textbf{El problema del árbol generador del peso mínimo} }
    $$ x_{e} = 1\ \text{si la arista $ e $ pertenece al árbol $ \forall e \in E $} $$

    $$ min\ \sum\limits_{e \in E}^{} l_{e}x_{e} $$
    $$ \left\{
    \begin{array}{llcc}
        s.a. & x_{e} \in \{0,1\} \forall e \in E & \\
        & \sum\limits_{e \in E}^{}x_{e} = n-1 & \\
        & \sum\limits_{e \in E(V ^{3})}^{} x_{e} \leq 2 & \forall V ^3 \subset V & |V^3| = 3\\
        & \sum\limits_{e \in E(V ^{4})}^{} x_{e} \leq 3 & \forall V ^4 \subset V & |V^4| = 4\\
        &  \vdots\\
        & \sum\limits_{e \in E(S}^{} x_{e} \leq |S|-1 & \forall S \subset V & 3 \leq |S| \leq n-1\\
        
    \end{array}
    \right. $$

    Haremos ahora la llamada formulación MTZ, que utiliza ''una especie de árbol dirigido'' comenzamos definiendo las variables:

    %TODO añadir un dibujo (está copiado 22/10)

    $$ u_i = \text{algo parecido al nivel del nodo $ i $ en la arborescencia} $$
    $$ x_{ij} = \left\{
    \begin{array}{ll}
        1 & \text{si $ i $ es el predecesor inmediato de $ j $ en el árbol con raíz en 1}\\
        0 & oc 
    \end{array}
    \right. $$

    $$ \left\{
    \begin{array}{llcc}
        min & \sum\limits_{i}^{}\sum\limits_{j=(i,j) \in E}^{} l_{ij}x_{ij}\\
        s.a. & x_{ij} \in \{0,1\} & \forall i,j: (i,j) \in E\\
        & \sum\limits_{i}^{}x_{ij} = 1 & \forall j \ne 1\\
        & x_{ij} + x_{ji} \leq 1 & \forall (i,j) \in E\\
        & u_i \in \mathbb{Z}^{+} & \forall i\\
        & u_1 = 0\\
        & u_j \geq u_i +1 - M (1-x_{ij}) & \forall i,j : (i,j) \in E & (x_{ij}=1 \implies u_j \geq u_i +1)

    \end{array}
    \right. $$

    Podemos cambiar la $ M $ por $ n-1 $ ya que $ u_j \leq n-1\ \forall j $, creando una mejor formulación del problema.

\end{exercise}


\begin{exercise}
    {\color{turquoise} \textbf{El problema del camino más corto entre dos vértices} $ s $ \textbf{ y } $ j $} 

    Usaremos longitudes no negativas, y sea $ x_{ij} = 1  $ si el camino atraviesa el nodo $ i $ y a continuación el $ j $.
    $$ \left\{
    \begin{array}{llrr}
        Min & \sum\limits_{i \in V}^{}\sum\limits_{j \in V: (i,j) \in E}^{} l_{ij}x_{ij}\\
        s.a. & x_{ij} \in \{0,1\} & \forall i,j: (i,j) \in E\\
        & \sum\limits_{j:(j,s) \in E}^{}x_{sj} = 1 \\
        & x_{js} = 0 & \forall j : (j,s) \in E\\
        & \sum\limits_{j:(j,t) \in E}^{} x_{jt} = 1\\
        & x_{tj} = 0 & \forall j : (j,t) \in E\\
        & \sum\limits_{i:(i,j) \in E}^{} x_{ij} = \sum\limits_{i:(i,j) \in E}^{} x_{ji} & \forall j \ne s,t & \text{Para todo nodo por el que entres, sales}
    \end{array}
    \right. $$
    Podemos ver que la tercera y la cuarta restricción no son necesarias ya que se obtienen de las otras.

    Veamos ahora otra formulación, si tenemos la estructura de árbol con raíz en $ s $ que contiene los caminos más cortos, supongamos que tenemos que enviar canicas desde la raíz de forma que llegue una a cada nodo. Las variables serán las canicas que pasan por cada arista.

    $$ x_{ij} = \text{nº de items (canicas) que circulan desde $ i $ hasta $ j $} \equiv $$$$ \equiv\text{nº de caminos más cortos desde $ s = 1 $ que contienen el subcamino $ i,(i,j),j $}$$


    $$ \left\{
    \begin{array}{llr}
        min & \sum\limits_{i}^{} \sum\limits_{j:(i,j)\in E}^{} l_{ij}x_{ij}\\
        s.a. & \sum\limits_{j:(1,j) \in E}^{} x_{1j} = n-1 \\
        & \sum\limits_{j: (i,j)}^{} x_{ij} = \sum\limits_{j:(j,i) \in E}^{} x_{ji}-1 & \forall i \ne 1\\
        & x_{ij},x_{ji} \in \mathbb{Z} ^{+} & \forall (i,j) \in E i<j
    \end{array}
    \right. $$

    Vamos a modificar esta formulación un poco para obtener otra equivalente que tendrá una matriz de restricciones totalmente unimodular.

    La restricción 1 la podemos intercambiar por $ \sum\limits_{j}^{}x_{1j}-x_{j_1} = n-1 $, entonces esta restricción $ R_1 = - \sum\limits_{i=2}^{n} R_i $, con lo que la podemos eliminar. Ahora, por un razonamiento análogo al visto en teoría para la matriz de incidencia del grafo bipartito, $ A $ es $ TU $ y, como $ b $ es entero, podemos eliminar la restricción de integridad. Es equivalente solucionar su problema dual:

    $$ \left\{
    \begin{array}{llr}
        Max & \sum\limits_{j\ne_1}^{}d_j\\
        s.a. & d_j \leq l_{1j} & \forall j:(1,j) \in E\\
        & d_j-d_i \leq l_{ij} & \forall i, \forall j \ne 1\ : (i,j) \in E
    \end{array}
    \right. $$
    

\end{exercise}






\chapter{Caminos más cortos. Recorridos por artistas y vértices}

%Chito, supongo que esto lo tendrás copiado, si no se copia en un momento.

%def de longitud de un camino

%El problema del camino más corto

%lema del subcamino mas corto, sigue la demostracion

\begin{demonstration}
    Si hubiera un camino más corto $ P_1 $ entre $ v_i $ y $ v_j $ que el subcamino $ P_2 $ entre $ v_i,v_j $, reemplazando $ P_2 $ por $ P_1 $ obtenemos o bien 1. o 2.:
    \begin{enumerate}
        \item Un camino más corto que el camino más corto con lo que tenemos una contradicción.
        \item Un paseo que contiene ciclos, la eliminación de estos ciclos nos deja un camino más corto que el camino más corto, de nuevo una contradicción.
    \end{enumerate}
\end{demonstration}

%teorema con el vector d, esta es la dem

\begin{demonstration}
    $ \impliedby $

    Supongamos $ d_j > d_i + l_{ij} $, entonces, podemos crear un camino más corto a $ j $ yuxtaponiendo el camino a $ i $ y la arista que une $ i $ con $ j $ si no se forman ciclos, si se formaran, basta con quitarlo y aún así tendríamos un camino más corto a $ j $. En ambos casos tenemos una contradicción.
    
    $ \implies $

    Sea $ j \in V $ cualquiera, sea $ P $ un camino cualquiera de $ s $ a $ j $, ¿se cumple que $ l(p) \geq d_j $? Si $ P = (s = i_0,i_1,...,i_{q}=j) $ tenemos que:

    $$ d_{i_1}-d_{i_0} \leq l_{i_0i_1} \hspace{5mm} d_{i_2}-d_{i_1} \leq l_{i_1i_2} \hspace{5mm} ...$$

    Sumando todo obtenemos que $d_j-\underbrace{d_{s}}_{0} = d_{i_q}-d_{i_0} \leq \sum\limits_{k}^{}l_{i_{k}i_{k+1}} = l(P)$

\end{demonstration}

%el siguiente con el vector d

%dem dijkstra

\begin{demonstration}
    Sea $ P $ camino entre $ s $ y $ j  $, ¿ $ l(p)\geq d_j $ ? De este camino $ (v_{s},v_{a},v_{b},...,v_{j}) $ sabemos que:
    \begin{itemize}
        \item $ v_{s} \not \in V' $
        \item $ v_j \in V' $
    \end{itemize}
    Si $ P_{2} $ es el subcamino desde $ s $ hasta el primer nodo de $ V' $ con último coste $ (v_{i_1},v_{i_2}) \in w(V') $, entonces:
    $$ l(P) \geq l(P_2) =\underbrace{l(P_1)}_{\text{longitud del subcamino que une $ s $ y $ v_{ij} $}}+l_{i_1i_2} \geq d_{i_1} + l_{i_1}l_{i_2} \geq _{(c)} d_{i_2} \geq d_j$$
\end{demonstration}



%grafo euleriano. Definicion de tour y grafo euleariano

%teorema grafo conexo euleriano sii todo los vertices son pares

\begin{demonstration}
    
    $ \implies $
    
    Como $ G $ es conexo, $ |g(v)| \geq 1 \forall v$, el paseo no repite aristas y atraviesa cada nodo añadiéndole grado 2 hasta cerrarse por lo que todos los nodos tienen grado par.

    $ \impliedby $
    
    Iniciamos un tour en $ v_1, (v_1,v_2)$ (la arista existe porque el grafo tiene que ser conexo). Como $ g(v_2) = \dot{2},\ \exists v_3,\ (v_2,v_3) \in E$, así podemos construir la sucesión $ (v_1,v_2),(v_2,v_3),(v_3,v_4),...,(v_i,v_{k}) $. Solo se detiene el proceso si se encuentra la arista $ v_{t},v_1 $ y no existen más aristas incidentes en $ v_1 $ que no estén en el paseo. En este caso pueden haber pasado dos cosas:

    \begin{enumerate}
        \item Si ya hemos recorrido todas las aristas hemos terminado.
        \item No hemos recorrido todas las aristas. Como $ G $ es conexo, $ \exists $ arista que no está en el paseo $ (v_{s},v_{x}) $ con $ v_{s} $ en el paseo. En este caso, podemos empezar un nuevo paseo empezando en $ v_{s} $ formado por la concatenación del que hemos formado antes reordenado para que empiece y acabe en el $ v_{s} $ y el que se genera de la misma forma que antes pero comenzando en $ (v_{s},v_{x}) $. Iteramos hasta agotar las aristas
    \end{enumerate}

\end{demonstration}

%algoritmo de Heissenbergz y el de Fleury

%definicion problema del cartero chino

%algoritmo de dancing

%%%%%%METODO PARA RESOLVER EL PROBLEMA DEL CARTERO CHINO

%usar alg de dancing

%calclar el coste de todos los emparejamientos de nodos impares

%seleccionar el emparejamiento de menor coste

%duçplicar todas las aristas perteneceientes a los caminos mas cortos para el emparejamiento elegido

%en el multigrafo euleraino obtenido, encontrar un ciclo euleriano usando el alg de Fleury


%Definicion de ciclo hamiltoniano y camino hamiltoniano
%def grafo clausura

\begin{theorem}
    {\color{turquoise} \textbf{Bondy-Chvatal}}
\end{theorem}

\begin{demonstration}

    $ \implies $

    Directo.

    $ \impliedby $

    Reducción al absurdo: supongamos que existe un grafo no hamiltonianos cuya clausura sí es hamiltoniana.
    
    Sea $ G_0,...,G_k = [G] $ con $ j $ el primer índice tal que $ G_{j} $ no es hamiltoniano pero $ G_{j+1} $ sí lo es. La última arista añadida, digamos que es $ (v_1,v_n) \not \in E_j $ y $ (v_1,v_n) \in E_{j+1} $, tiene que estar en el ciclo hamiltoniano de $ G_{j+1} $. Sea este ciclo $ v_1,...,v_n,v_1 $.
    
    Retirando $ (v_1,v_n) $ de nuevo, se sigue que en $ G_j $ existía un camino hamiltoniano $ (v_1,v_2,...,v_n) $.
    
    Sean $ Y = \{i \in \{3,...,n-1\} : (v_1,v_i) \in E_j\} $ (algunos de los vecinos de $ v_1 $) y $ X = \{i \in \{3,...,n+1\}:(v_n,i) \in E_j\} $ (los nodos a la derecha de algunos de los vectores de $ v_n $). Entonces $ |Y| = g_{G_j}(v_1) -1$ y $ |X| = g_{Gj}(v_n)-1 $ y $ |X| + |Y| \geq_{\text{por la contrucción de $ G_{j+1} $}} n-2  $.

    Como $ X,Y \subset \{3,4,...,n-1\}_{\text{con cardinal $ n-3 $}} $ y sus cardinales suman $ n-2 \implies X \cap Y \ne \emptyset \implies \exists v_{k} \in X \cap Y \implies $
    $$ \implies \left\{
    \begin{array}{l}
        (v_{k},v_1) \in E_j\\
        (v_{k-1},v_n) \in E_j
    \end{array}
    \right. $$
    Y tenemos que $ (v_1,..,v_{k-1},v_{n},v_{n-1},...,v_{k},v_{1}) $ es un ciclo hamiltoniano.
\end{demonstration}


%teorema de chavtal

\begin{demonstration}
    
    $ \implies $

    Utilizaremos el contrarrecíproco, si tenemos un $ i < \dfrac{n}{2} $ tal que $ a_i \leq 1 $ y $ a_{n-1} < n-1 $

    %dibujo 4/11

    $ \impliedby $

    Supongamos que no es hamiltoniano. Sea $ G(V,E) $ un grafo con el mayor número posible de aristas que no es hamiltoniano con grafos $ g_{1}\leq...\leq g_n $. Esta grafo satisface la condición 

    % $$ g_i \leq i \implies a_i\leq i \implies a_{n-i} \leq n-1 \implies g_{n-i} \geq n-i \forall i < n/2 $$
    % Sean $ u,v \in V $m con $ (u,v) \not \in E $ y $ g(u)+g(v) $ lo mayor posible.

    $$ g_i \leq i \implies a_i \leq g_i \leq 1 \implies a_{n-i} \geq n-1 \implies g_{n-i} \geq a_{n-i} \geq n-i $$
    
    De todos los $ G $ en estas condiciones elegimos uno con número máximo de aristas. Sean $ u,v \in V $ con $ (u,v) \not \in E $ con máxima suma de $ g(u)+g(v) $. Sea $ u $ el de menor grado: $ g(u) \leq g(v) $. Como $ G $ es maximal, al añadir la arista $ (u,v) $ pasa a ser hamiltoniano, luego $ \exists $ camino hamiltoniano $ u \sim v $ en $ G $:
    $$ u = v_1-v_2-...-v_n = v $$
    
    Como vimos ayer si $ u $ es vecino de $ v_{i} $, no puede darse que $ v $ sea vecino de $ v_{j} $, $ j<i $. 

    Consideremos el conjunto de vecinos de $ u $ ($ N(u) \subset \{v_1,...,v_{n-1}\} $) y el conjunto de nodos de $ v_i $ que preceden inmediatamente en el camino a un vecino de $ u\subset \{v_1,..,v_{n-1}\} $. La existencia de un nodo en la intersección de estos 2 conjuntos implica la existencia de un ciclo ham. en $ G $, luego son disjuntos.
    
    La unión de estos dos conjuntos está incluida en $ \{v_1,...,v_{n-1}\} $. Uno de los conjuntos tiene cardinal $ g(u) $ y el otro $ g(v) $, como la unión es disjunta el cardinal es $ g(u) + g(v) $. Por tanto, tenemos que $ g(u)+g(v) < n \implies g(u) < \dfrac{n}{2} $

    Sea $ x \in V $ un nodo que precede en el camino a un vecino de $ u $ (y que por tanto no es vecino de $ v $). Como tomamos $ u,v $ que no fueran vecinos con máximo $ g(u)+g(v) $, tenemos que $ g(x) \leq g(u) $. En las condiciones de $ x $ tengo tantos nodos como $ |N(u)| = g(u) $. Sea $ i = g(u) $, entonces $ i<\dfrac{n}{2}, g(i) \leq i \implies g_{n-i} \geq n-i $

    Luego al menos hay  $ i+1 $ nodos  con grado  $ \geq n-i $. Como $ i = g(u) $ no todos estos nodos pueden ser vecinos de $ n $. Sea $ w \not \in N(u) $ con hgrado $ \geq n-i $.
    $$ g(w) \geq n-g(u)\ \implies \text{Contradicción con la elección de $ v $} $$

    %definicion de problema del viajante de comercio

\end{demonstration}

\chapter{Coloración}

%%formula de euler

\begin{demonstration}
    $ G $ es conexo planar.
    
    Si $ G $ es bosque, entonces $ c = 1 $, luego $ n-m+c = n-(n-1)+1=2 $.
    
    Si $ G $ no es un bosque, aplicamos inducción. Si $ c\geq 2 $:
    Habrá una arista $ e $ que no sea puente, la retiramos para obtener un nuevo grafo con $ m = m-1 $ aristas y $ c' = c-1 $ caras (al formarse 2 caras a cuyos entornos pertenecía $ e $) $ \implies $ aplicamos inducción.

\end{demonstration}

%%lema des m
%perdone maestro me reperdí
% \begin{demonstration}
%     $ G $ planar y $ n \geq 3 \implies m\leq 3n-6$

%     Si $ m = 1,2 $ es trivial, si $ m $ es más grande:

%     \begin{itemize}
%         \item Al menos hay 3 aristas en el contorno de una cara
%         \item A lo sumo una arista pertenece a 2 caras
%             $$ 3c \leq 3m $$
%     \end{itemize}

% \end{demonstration}

\end{document}